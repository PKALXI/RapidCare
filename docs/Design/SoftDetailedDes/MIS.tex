\documentclass[12pt, titlepage]{article}

\usepackage{amsmath, mathtools}

\usepackage[round]{natbib}
\usepackage{amsfonts}
\usepackage{amssymb}
\usepackage{graphicx}
\usepackage{colortbl}
\usepackage{xr}
\usepackage{hyperref}
\usepackage{longtable}
\usepackage{xfrac}
\usepackage{tabularx}
\usepackage{float}
\usepackage{siunitx}
\usepackage{booktabs}
\usepackage{multirow}
\usepackage[section]{placeins}
\usepackage{caption}
\usepackage{fullpage}

\hypersetup{
bookmarks=true,     % show bookmarks bar?
colorlinks=true,       % false: boxed links; true: colored links
linkcolor=red,          % color of internal links (change box color with linkbordercolor)
citecolor=blue,      % color of links to bibliography
filecolor=magenta,  % color of file links
urlcolor=cyan          % color of external links
}

\usepackage{array}

\externaldocument{../../SRS/SRS}

\input{../../Comments}
\input{../../Common}

\newcommand{\projname}{RapidCare}

\begin{document}

\title{Module Interface Specification for \progname{}}

\author{\authname}

\date{\today}

\maketitle

\pagenumbering{roman}

\section{Revision History}

\begin{tabularx}{\textwidth}{p{3cm}p{2cm}X}
\toprule {\bf Date} & {\bf Version} & {\bf Notes}\\
\midrule
Jan 14, 2025 & 1.1 & Initial Document\\
Jan 17,2025 & 1.2 & Revised Document Incorporating Feedback\\

\bottomrule
\end{tabularx}

~\newpage

\section{Symbols, Abbreviations and Acronyms}

See SRS Documentation at \href{https://github.com/Inreet-Kaur/capstone/blob/main/docs/SRS/SRS.pdf} {SRS document}\\
\\


\begin{tabularx}{\textwidth}{p{2cm}X}
  \toprule
  {\bf Symbol} & {\bf Description}                                                                                     \\
  \midrule
  MG           & Module Guide                                                                                          \\
  M            & Module                                                                                                \\
  MIS          & Module Interface Specification                                                                        \\
  API          & Application Programming Interface                                                                     \\
  MFA          & Multi-Factor Authentication                                                                           \\
  \bottomrule
\end{tabularx}

\newpage

\tableofcontents

\newpage

\pagenumbering{arabic}

\section{Introduction}

The following document details the Module Interface Specifications for the \projname application.

Complementary documents include the System Requirement Specifications and Module Guide. The full documentation and implementation can be found at \url{https://github.com/PKALXI/RapidCare/blob/main/docs/Design/SoftArchitecture/MG.pdf}.

\section{Notation}

% \wss{You should describe your notation.  You can use what is below as
%   a starting point.}

The structure of the MIS for modules comes from \citet{HoffmanAndStrooper1995},
with the addition that template modules have been adapted from
\cite{GhezziEtAl2003}.  The mathematical notation comes from Chapter 3 of
\citet{HoffmanAndStrooper1995}.  For instance, the symbol := is used for a
multiple assignment statement and conditional rules follow the form $(c_1
\Rightarrow r_1 | c_2 \Rightarrow r_2 | ... | c_n \Rightarrow r_n )$.

\subsection{Primitive Data Types}

The following table summarizes the primitive data types used by \projname. 

\begin{center}
\renewcommand{\arraystretch}{1.2}
\noindent 
\begin{tabular}{l l p{7.5cm}} 
\toprule 
\textbf{Data Type} & \textbf{Notation} & \textbf{Description}\\ 
\midrule
character & char & a single symbol or digit\\
integer & int & a number without a fractional component in (-$\infty$, $\infty$) \\
real & $\mathbb{R}$ & any number in (-$\infty$, $\infty$)\\
boolean & boolean & value of true or false \\
\bottomrule
\end{tabular} 
\end{center}

\noindent
The specification of \projname \ uses some derived data types: sequences, strings, and
tuples, maps. Sequences are lists filled with elements of the same data type. Strings
are sequences of characters. Tuples contain a list of values, potentially of
different types. Maps contain key-value pairs. In addition, \projname \ uses functions, which
are defined by the data types of their inputs and outputs. Local functions are
described by giving their type signature followed by their specification.

\subsection{Imported Data Types}

\begin{center}
\renewcommand{\arraystretch}{1.2}
\noindent 
\begin{tabular}{l l p{7.5cm}} 
\toprule 
\textbf{Data Type} & \textbf{Notation} & \textbf{Description}\\ 
\midrule
FormData & FormData & A built-in browser API object used to construct a set of key/value pairs representing form fields and their values for HTTP requests. The keys and values are arbitrary to the use case.\\
TensorFlow Sequential Model & tf.sequential & A deep learning model architecture from TensorFlow that allows layers to be stacked sequentially\\
\bottomrule
\end{tabular} 
\end{center}

\section{Module Decomposition}

The following table is taken directly from the Module Guide document for this project.

\begin{table}[h!]
\centering
\begin{tabular}{p{0.3\textwidth} p{0.6\textwidth}}
\toprule
\textbf{Level 1} & \textbf{Level 2}\\
\midrule
{Hardware-Hiding} & None \\
\midrule
\multirow{7}{0.3\textwidth}{Behaviour-Hiding} & User Authentication Module\\
& Administrator View Module\\
& Patient View Module\\
& Administrator Model Module\\
& Patient Model Module\\
& Broker Module\\
& Administrator Account Management Module\\
& Patient Account Management Module\\
\midrule
\multirow{3}{0.3\textwidth}{Software Decision} & Transcription Module\\
& Classification Module\\
& Diagnosis Prediction Module\\
& Medicine Prediction Module\\
\bottomrule
\end{tabular}
\caption{Module Hierarchy}
\label{TblMH}
\end{table}

\newpage

\section{MIS of User Authentication Module} \label{Module_UserAuth}

\subsection{Module}
UserAuthentication

\subsection{Uses}
Firebase Auth\\

\subsection{Syntax}

\subsubsection{Exported Constants}
isAuthenticated : boolean

\subsubsection{Exported Access Programs}

\begin{center}
\begin{tabular}{p{3cm} p{4cm} p{3cm} p{3.5cm}}
\hline
\textbf{Name} & \textbf{In} & \textbf{Out} & \textbf{Exceptions} \\
\hline
Auth & - & React.component &  RenderError\\
\hline
\end{tabular}
\end{center}

\subsection{Semantics}

\subsubsection{State Variables}
isUserAdmin : boolean

\subsubsection{Environment Variables}
N/A

\subsubsection{Assumptions}
N/A

\subsubsection{Access Routine Semantics}

\noindent Auth():
\begin{itemize}
\item transition: Renders the login page on the screen.
\item output: N/A
\item exception: RenderError — Thrown if the component fails to render.
\end{itemize}

\subsubsection{Local Functions}

\noindent login():
\begin{itemize}
\item transition: Renders adminstrator view page if user is administrator otherwise renders patient view page. 
\item output: N/A
\item exception: InvalidCredentials - Thrown if the user enters invalid credentials.
\end{itemize}

\noindent ResetPassowrd():
\begin{itemize}
\item transition: Sends a reset link to the provided email and renders the login page.
\item output: N/A
\item exception: InvalidInpuError - Thrown if user input is invalid or unregistered email.
\end{itemize}

\newpage


\section{MIS of Administrator View Module} \label{Module_ AdminView}

\subsection{Module}
Administrator

\subsection{Uses}
\hyperref[Module_Broker]{Broker Module} \\
ReactJS\\

\subsection{Syntax}

\subsubsection{Exported Constants}
N/A

\subsubsection{Exported Access Programs}

\begin{center}
\begin{tabular}{p{3cm} p{2cm} p{4cm} p{3cm}}
\hline
\textbf{Name} & \textbf{In} & \textbf{Out} & \textbf{Exceptions} \\
\hline
Administrator & - & React.component &  RenderError\\
\hline
\end{tabular}
\end{center}

\subsection{Semantics}

\subsubsection{State Variables}
isAuthenticated : boolean

\subsubsection{Environment Variables}
Screen interface\\
Keyboard\\
Microphone

\subsubsection{Assumptions}
User has a functional screen, keyboard, and microphone.

\subsubsection{Access Routine Semantics}

\noindent Administrator():
\begin{itemize}
\item transition: Renders a react component of the administrator view page.
\item output: N/A
\item exception: RenderError — Thrown if the component fails to render.
\end{itemize}

\subsubsection{Local Functions}

\noindent handleAdminAccount(id : String, record : FormData, requestType : String):
\begin{itemize}
\item transition: Sends an API request to the Broker Module to process an add, delete, or update operation in the Administrator Database.
\item output: N/A
\item exception: InvalidInputError - Thrown if the formData is missing a field or is invalid.
\end{itemize} 

\noindent validateInput(InputField : String):
\begin{itemize}
\item transition: Renders a success or error message outlining the action performed.
\item output: N/A
\item exception: InvalidInputError - The input data is incomplete or invalid.
\end{itemize}

\newpage

\section{MIS of Patient View Module} \label{Module_PatientView}

\subsection{Module}
Patient

\subsection{Uses}
\hyperref[Module_Broker]{Broker Module} \\
ReactJS\\

\subsection{Syntax}

\subsubsection{Exported Constants}
N/A

\subsubsection{Exported Access Programs}

\begin{center}
  \begin{tabular}{p{3cm} p{2cm} p{4cm} p{3cm}}
\hline
\textbf{Name} & \textbf{In} & \textbf{Out} & \textbf{Exceptions} \\
\hline
Patient & - & React.component & RenderError \\
\hline
\end{tabular}
\end{center}

\subsection{Semantics}

\subsubsection{State Variables}
isAuthenticated : boolean

\subsubsection{Environment Variables}
Screen interface\\
Keyboard\\
Microphone\\

\subsubsection{Assumptions}
User has a functional screen, keyboard, and microphone.

\subsubsection{Access Routine Semantics}

\noindent Patient():
\begin{itemize}
\item transition: Renders a react component of the patient view page.
\item output: N/A
\item exception: RenderError — Thrown if the component fails to render.
\end{itemize}

\subsubsection{Local Functions}

\noindent handlePatientAccount(id : String, record : FormData, requestType : String):
\begin{itemize}
\item transition: Sends an API request to the API Module to process an add, delete, or update operation in the Patient Database.
\item output: N/A
\item exception: InvalidInputError - Thrown if the formData is missing a field or is invalid.
\end{itemize}

\noindent validateInput(InputField : String):
\begin{itemize}
\item transition: Renders a success/error message outlining the action performed.
\item output: N/A
\item exception: InvalidInputError - The input data is incomplete or invalid.
\end{itemize}

\newpage

\section{MIS of Broker Module } \label{Module_Broker}


\subsection{Module}
Broker

\subsection{Uses}

\begin{itemize}
  \item \hyperref[Transcription_Module]{Transcription Module}
  \item \hyperref[Classification_Module]{Classification Module}
  \item \hyperref[diag_pred_mod]{Diagnosis Prediction Module}
  \item \hyperref[med_pred_mod]{Medicine Prediction Module}
  \item \hyperref[Module_AdminAccount]{Administrator Account Management Module}
  \item \hyperref[Module_PatientAccountManag]{Patient Account Management Module}
\end{itemize}

\subsection{Syntax}

\subsubsection{Exported Constants}
N/A

\subsubsection{Exported Access Programs}

\begin{center}
\begin{tabular}{p{4cm} p{3cm} p{4cm} p{3.5cm}}
\hline
\textbf{Name} & \textbf{In} & \textbf{Out} & \textbf{Exceptions} \\
\hline
\texttt{getToken} & - & token : String & InvalidTokenError \\
\texttt{transcribeText} & request : FormData & transcribedText : String & FailedResponseError \\
\texttt{classifyText} & request : FormData & classifiedText : map  & FailedResponseError \\
\texttt{predictDiagnosis} & request : FormData & applicableDiagnosis : String & FailedResponseError \\
\texttt{predictedMedicine} & request : FormData & applicableMedicine : String & FailedResponseError \\
\hline
\end{tabular}
\end{center}

\subsection{Semantics}

\subsubsection{State Variables}

\begin{itemize}
  \item secretKey : String 
\end{itemize}

\subsubsection{Environment Variables}

N/A

\subsubsection{Assumptions}

\begin{itemize}
  \item Requires a stable database connection.
  \item All end points in distributed systems are up.
\end{itemize}

\subsubsection{Access Routine Semantics}

\noindent getToken():
\begin{itemize}
    \item \textbf{Transition:} N/A
    \item \textbf{Output:} Issues an access token with encoded with the \texttt{secretKey}
    \item \textbf{Exception:} InvalidTokenError : Invalid or expired authorization code.
\end{itemize}

\noindent transcribeText(request : FormData):
\begin{itemize}
    \item \textbf{Transition:} N/A
    \item \textbf{Output:} Check if request header has valid token with \texttt{authorize}, then return live transcription of audio bytes in the request.body.
    \item \textbf{Exception:} FailedResponseError: The corresponding service/module has returned an error.
\end{itemize}

\noindent classifyText(request : FormData):
\begin{itemize}
    \item \textbf{Transition:} N/A
    \item \textbf{Output:} Check if request header has valid token with \texttt{authorize}, then classify the request.text into the fields given in request.chart which represents the medical chart data in FormData form. Return map of text classified. 
    \item \textbf{Exception:} FailedResponseError: The corresponding service/module has returned an error.
\end{itemize}

\noindent predictDiagnosis(request : FormData):
\begin{itemize}
    \item \textbf{Transition:} N/A
    \item \textbf{Output:} Check if request header has valid token with \texttt{authorize}, then provide stream of diagnosis suggestions based on request.chart which represents the medical chart data in FormData form.
    \item \textbf{Exception:} FailedResponseError: The corresponding service/module has returned an error.
\end{itemize}


\noindent predictMedicine(request : FormData):
\begin{itemize}
    \item \textbf{Transition:} N/A
    \item \textbf{Output:} Check if request header has valid token with \texttt{authorize}, then provide stream of medicine suggestions based on request.chart which represents the medical chart data in FormData form.
    \item \textbf{Exception:} FailedResponseError: The corresponding service/module has returned an error.
\end{itemize}

\subsubsection{Local Functions}

\noindent authorize(header : String):
\begin{itemize}
    \item \textbf{Transition:} N/A
    \item \textbf{Output:} This is a header function to make sure the request is authorized, here this function returns whether the request is authorized.
    \item \textbf{Exception:} Invalid client credentials.
\end{itemize}

\newpage

\section{MIS of Administrator Model Module } \label{Admin_Model_Module}

\subsection{Module}

AdminModel

\subsection{Uses}

N/A

\subsection{Syntax}

\subsubsection{Exported Constants}

N/A

\subsubsection{Exported Access Programs}

\begin{center}
\begin{tabular}{p{3cm} p{4cm} p{4cm} p{3.5cm}}
\hline
\textbf{Name} & \textbf{In} & \textbf{Out} & \textbf{Exceptions} \\
\hline
AdminModel & - & - & - \\ 
\hline
\end{tabular}
\end{center}

\subsection{Semantics}

\subsubsection{State Variables}

\begin{itemize}
    \item name : String
    \item age : int
    \item location : String
    \item profession : String
\end{itemize}

\subsubsection{Environment Variables}

N/A

\subsubsection{Assumptions}

N/A

\subsubsection{Access Routine Semantics}

\noindent getter(): This is the boiler state of this variable where it will get a certain start and return it.
\begin{itemize}
    \item transition: N/A
    \item output: The output depends on the data type of the parameter. It returns the current value of the requested data element.
    \item exception: N/A
\end{itemize}

\noindent setter(): This is the boiler state of this variable where it will get a certain start and return it.
\begin{itemize}
    \item transition: Updates the internal state of the data model, either by adding or updating data. This also changes certain state variables.
    \item output: N/A
    \item exception: N/A
\end{itemize}

\subsubsection{Local Functions}

\noindent init(inputField: String):
\begin{itemize}
\item transition: N/A
\item output: Necessary data structures and connections are made to manage and access data.
\item exception: N/A
\end{itemize}

\newpage

\section{MIS of Patient Model Module } \label{Patient_Model_Module}

\subsection{Module}

PatientModule

\subsection{Uses}

N/A

\subsection{Syntax}

\subsubsection{Exported Constants}

N/A

\subsubsection{Exported Access Programs}

\begin{center}
\begin{tabular}{p{3cm} p{4cm} p{4cm} p{3.5cm}}
\hline
\textbf{Name} & \textbf{In} & \textbf{Out} & \textbf{Exceptions} \\
\hline
PatientModel & - & - & - \\ 
\hline
\end{tabular}
\end{center}

\subsection{Semantics}

\subsubsection{State Variables}

\begin{itemize}
    \item patientName : String
    \item dateOfBirth : date
    \item age : int
    \item gender : String
    \item address : String
    \item email : String
    \item contactNumber : String
    \item allergies : String
    \item medicalHistory : String
    \item medications : String
    \item insuranceInfo : String
\end{itemize}

\subsubsection{Environment Variables}

N/A

\subsubsection{Assumptions}

N/A

\subsubsection{Access Routine Semantics}

\noindent getter(): This is the boiler state of this variable where it will get a certain start and return it.
\begin{itemize}
    \item transition: N/A
    \item output: The output depends on the data type of the parameter. It returns the current value of the requested data element.
    \item exception: N/A
\end{itemize}

\noindent setter(): This is the boiler state of this variable where it will get a certain start and return it.
\begin{itemize}
    \item transition: Updates the internal state of the data model, either by adding or updating data. This also changes certain state variables
    \item output: N/A
    \item exception: N/A
\end{itemize}

\noindent init(inputField : String):
\begin{itemize}
\item transition: N/A
\item output: Necessary data structures and connections are made to manage and access data.
\item exception: N/A
\end{itemize}

\subsubsection{Local Functions}

N/A

\newpage

  
\section{MIS of Transcription Module} \label{Transcription_Module}

\subsection{Module}
TranscriptionModule

\subsection{Uses}

N/A

\subsection{Syntax}

\subsubsection{Exported Constants}

N/A

\subsubsection{Exported Access Programs}

\begin{center}
\begin{tabular}{p{3cm} p{4cm} p{4cm} p{3cm}}
\hline
\textbf{Name} & \textbf{In} & \textbf{Out} & \textbf{Exceptions} \\
\hline
Transcription & audioData : byte[] & transcribedText : String & InvalidInputError \\ 
\hline
\end{tabular}
\end{center}

\subsection{Semantics}

\subsubsection{State Variables}

N/A

\subsubsection{Environment Variables}

N/A

\subsubsection{Assumptions}

N/A

\subsubsection{Access Routine Semantics}

\noindent transMod(audioData : byte[]): 
\begin{itemize}
  \item transition: N/A
  \item output: Transcribed text transcribed from the audio bytes. 
  \item exception: InvalidInputError - If the bytes could not be converted to text. 
\end{itemize}

\subsubsection{Local Functions}

N/A

~\newpage
  
\section{MIS of Classification Module} \label{Classification_Module} 

\subsection{Module}

ClassificationModule

\subsection{Uses}

N/A

\subsection{Syntax}

\subsubsection{Exported Constants}

N/A

\subsubsection{Exported Access Programs}

\begin{center}
  \begin{tabular}{p{2.5cm} p{4cm} p{4cm} p{3cm}}  % Adjusted last column width
  \hline
  \textbf{Name} & \textbf{In} & \textbf{Out} & \textbf{Exceptions} \\
  \hline
  Classification & transcribedText : String & classifiedText : String & - \\
  \hline
  \end{tabular}
  \end{center}

\subsection{Semantics}

\subsubsection{State Variables}

N/A

\subsubsection{Environment Variables}

N/A

\subsubsection{Assumptions}

\begin{itemize}
  \item It is assumed that the transcribed text is in English language. Let $T$ be the complete transcribed text ($T = (w_1, w_2, \ldots, w_n)$ where $n$ is a natural number). Let $E$ be the set of all valid English words. Therefore: $(\forall w \in T \cdot (w \in E))$
\end{itemize}

\subsubsection{Access Routine Semantics}

\noindent classifyModule(transcribedText : String):

\begin{itemize}
\item transition: N/A
\item output: 
  \begin{itemize}
    \item Let $C$ be the set of all possible classified texts. $C = (c_1, c_2, \ldots, c_n)$ where $n$ is a natural number
    \item Let $T$ be the set of all possible transcribed texts. $T = (t_1, t_2, \ldots, t_n)$ where $n$ is a natural number
    \item Then classifyModule returns a classification ``$c$'' for a transcribed text ``$t$'' such that it fulfills the relation $R$ where $R \subseteq T \times C$, where $(t, c) \in R$.
  \end{itemize}
\item exception: N/A 
\end{itemize}



\subsubsection{Local Functions}

N/A

\newpage

\section{MIS of Diagnosis Prediction Module} \label{diag_pred_mod}

\subsection{Module}

DiagnosisPred

\subsection{Uses}
\begin{itemize}
  \item Tensorflow
  \item Scikit-Learn
  \item Flask
  \item Flask-CORS
\end{itemize}

\subsection{Syntax}

\subsubsection{Exported Constants}
predictedDiagnosis : String \\

\subsubsection{Exported Access Programs}

\begin{center}
\begin{tabular}{p{3cm} p{4cm} p{4cm} p{3cm}}
\hline
\textbf{Name} & \textbf{In} & \textbf{Out} & \textbf{Exceptions} \\
\hline
diagnosePatient & request : FormData & Prediction of possible Diagnosis : String & InputDimError \\
\hline
\end{tabular}
\end{center}

\subsection{Semantics}

\subsubsection{State Variables}

\begin{itemize}
  \item model : tf.sequential
  % \item modelPipline : Abstract Object
\end{itemize}

\subsubsection{Environment Variables}

N/A

\subsubsection{Assumptions}

\begin{itemize}
  \item Patients are not making up any items in the patient chart and all input features are accurate.

\end{itemize}

\subsubsection{Access Routine Semantics}

\noindent diagnosePatient(request : FormData):
\begin{itemize}
\item transition: N/A
\item output: Returns the predicted diagnosis for the patient based on preprocessed input given by \texttt{preProcessData}. FormData is expected to contain the key-value format of (past medical history:String, symptoms:String, user characteristics including age:int and weight : int, and physician confirmed diagnosis:String)
\item exception: InputDimError - The expected request body items were not received or were in the wrong format.
\end{itemize}



\subsubsection{Local Functions}

\noindent preProcessData(pastHistory : String, symptoms : String, user):
\begin{itemize}
\item transition: N/A
\item output: Preprocess the text using ``Bag of Words'' (A String preprocessing technique) and normalize (scaling) the continuous inputs and return the preprocessed data as a map.
\item exception: InputDimError - The expected arguments were not received or were in the wrong format.
\end{itemize}

\newpage
  
\section{MIS of Medicine Prediction Module} \label{med_pred_mod}


\subsection{Module}

MedPred

\subsection{Uses}
\begin{itemize}
  \item Tensorflow
  \item Flask
  \item Flask-CORS
  \item Scikit-Learn

\end{itemize}

\subsection{Syntax}

\subsubsection{Exported Constants}
possibleMedicine : String\\

\subsubsection{Exported Access Programs}

\begin{center}
\begin{tabular}{p{3cm} p{4cm} p{4cm} p{3cm}}
\hline
\textbf{Name} & \textbf{In} & \textbf{Out} & \textbf{Exceptions} \\
\hline
medicatePatient & request : FormData & Prediction of possible Medicine : String & InputDimError \\
\hline
\end{tabular}
\end{center}

\subsection{Semantics}

\subsubsection{State Variables}

\begin{itemize}
  \item model : tf.sequential
  % \item modelPipeline : Abstract Object
\end{itemize}

\subsubsection{Environment Variables}

N/A

\subsubsection{Assumptions}


\begin{itemize}
  \item Patients are not making up any items in the patient chart and all input features are accurate.
  % \item Training pipeline for the model is setup
\end{itemize}

\subsubsection{Access Routine Semantics}

\noindent medicatePatient(request : FormData):
\begin{itemize}
\item transition: N/A
\item output: preprocess the request body (FormData) which contains the has the key-value format of (past medical history:String, symptoms:String, user characteristics including age:int and weightLint, and physician confirmed diagnosis:String)
\item exception: InputDimError - The expected request body items were not received or were in the wrong format.
\end{itemize}

\subsubsection{Local Functions}

\noindent preProcessData(diagnosis : String, pastHistory : String, symptoms : String, user):
\begin{itemize}
\item transition: N/A
\item output: Preprocess the text using ``Bag of Words'' (A String preprocessing technique) and normalize (scaling) the continuous inputs, finally the diagnosis will be encoded using a LabelEncoder then return the preprocessed data as a map.
\item exception: InputDimError - The expected arguments were not received or were in the wrong format.
\end{itemize}

\newpage

\section{MIS of Administrator Account Management Module} \label{Module_AdminAccount}

\subsection{Module}
AdministratorAccountManagement

\subsection{Uses}
N/A
\subsection{Syntax}

\subsubsection{Exported Constants}
N/A

\subsubsection{Exported Access Programs}

\begin{center}
\begin{tabular}{p{3.5cm} p{4cm} p{1cm} p{2.5cm}}
\hline
\textbf{Name} & \textbf{In} & \textbf{Out} & \textbf{Exceptions} \\
\hline
addHealthProf & id : String, value : FormData & - & InvalidInputError \\
deleteHealthProf & id : String & - & IdNotFound\\
updateHealthProf & id : String, value : FormData & - & InvalidInputError \\
\hline
\end{tabular}
\end{center}

\subsection{Semantics}

\subsubsection{State Variables}
dbConnection : Database connection point.

\subsubsection{Environment Variables}
N/A

\subsubsection{Assumptions}
Database containing the account information for healthcare professionals exists.

\subsubsection{Access Routine Semantics}

\noindent addHealthProf(id : String, record : FormData):
\begin{itemize}
\item transition: Adds a new document with the provided details to the database.
\item output: N/A
\item exception: InvalidInputError - The input data is incomplete or invalid or a duplicate document already exists.
\end{itemize}

\noindent deleteHealthProf(id : String):
\begin{itemize}
\item transition: Deletes the corresponding document from the database.
\item output: N/A
\item exception: IdNotFound - Provided id is invalid or does not exist.
\end{itemize}

\noindent updateHealthProf(id : String, record : FormData):
\begin{itemize}
\item transition: Update document with the provided details in the database.
\item output: N/A
\item exception: InvalidInputError - The input data is incomplete or invalid.
\end{itemize}

\subsubsection{Local Functions}

N/A

\newpage

\section{MIS of Patient Account Management Module} \label{Module_PatientAccountManag}

\subsection{Module}
PatientAccountManagement

\subsection{Uses}
N/A

\subsection{Syntax}

\subsubsection{Exported Constants}
N/A

\subsubsection{Exported Access Programs}

\begin{center}
\begin{tabular}{p{4cm} p{4.5cm} p{1cm} p{2.5cm}}
\hline
\textbf{Name} & \textbf{In} & \textbf{Out} & \textbf{Exceptions} \\
\hline
createPatientRecord & id : String, value : FormData & - & InvalidInputError \\
deletePatientRecord & id : String & - & IdNotFound\\
updatePatientRecord & id : String, value : FormData & - & InvalidInputError \\
\hline
\end{tabular}
\end{center}

\subsection{Semantics}

\subsubsection{State Variables}
dbConnection : Database connection point.

\subsubsection{Environment Variables}
N/A

\subsubsection{Assumptions}
Database containing the account information for the patients exists.

\subsubsection{Access Routine Semantics}

\noindent createPatientRecord(id : String, record : FormData):
\begin{itemize}
\item transition: Adds a new document with the provided details to the database.
\item output: N/A
\item exception: InvalidInputError - The input data is incomplete or invalid or a duplicate document already exists.
\end{itemize}

\noindent deletePatientRecord(id : String):
\begin{itemize}
\item transition: Deletes the corresponding document from the database.
\item output: N/A
\item exception: IdNotFound - Provided id is invalid or does not exist.
\end{itemize}

\noindent updatePatientRecord(id : String, record : FormData):
\begin{itemize}
\item transition: Update document with the provided details in the database.
\item output: N/A
\item exception: InvalidInputError - The input data is incomplete or invalid.
\end{itemize}

\subsubsection{Local Functions}

N/A

\newpage


\bibliographystyle {plainnat}
\bibliography {../../../refs/References}

% \newpage

% \section{Appendix} \label{Appendix}

% \wss{Extra information if required}

\newpage{}

\section*{Appendix --- Reflection}

% \wss{Not required for CAS 741 projects}

The information in this section will be used to evaluate the team members on the
graduate attribute of Problem Analysis and Design.

\input{../../Reflection.tex}

\begin{enumerate}
  \item What went well while writing this deliverable?\\
  This document has let us build more on the semantics and uses of each module by module decomposition. While going through the outline of this document, we were able to decompose semantics into different variables as well as assumptions that each module will have.

  \item What pain points did you experience during this deliverable, and how did you resolve them?\\
  Every team project has challenges that must be solved in order to move forward successfully. We had to create a plan to gurantee smooth operations. To ensure that all of our plans are in line with the system, we need to plan user-hierarchy diagram that complement our project. Along with this, we also needed to decide how the modules will be decomposed in the best way possible. In order to contribute to the document and evaluate each other’s work as effectively as possible, we also needed to set up a schedule. 

  \item Which of your design decisions stemmed from speaking to your client(s) or a proxy (e.g. your peers, stakeholders, potential users)? For those that were not, why, and where did they come from?\\
  After having conversation with our client, we created data model modules for both administrator and patient. In order to create state variables for both administrator and patient profile, our supervisor gave us a run down of the general attributes from which we selected the ones that are relevant to our system. Rest of the decision decisions came up by deciding as a team and were not stemmed from our client.  

  \item While creating the design doc, what parts of your other documents (e.g. requirements, hazard analysis, etc), it any, needed to be changed, and why?\\
  While creating this design document, we had to edit Software Requirement Specification (SRS) document to modify CI/CD implementation strategy from Jankins to GitHub actions. This is because unlike Jenkins, GitHub Actions offer excellent scalability and reliability for our CI/CD pipelines. It also offers some security features such that code scanning and vulnerability alerts. We also updated the Hazard Analysis documentation for modifying the data layer to include a medicine prediction database and diagnosis prediction database for the system to predict treatment based on the symptoms.  

  \item What are the limitations of your solution?  Put another way, given unlimited resources, what could you do to make the project better? (LO\_ProbSolutions)\\
  One of the limitations of this project is that if not trained properly, data model may provide incorrect prediction suggestions. If given unlimited resources, we would invest in some of the experts to prioritize easy interoperability and data exchange through connection with other healthcare systems. 

  \item Give a brief overview of other design solutions you considered.  What are the benefits and tradeoffs of those other designs compared with the chosen design?  From all the potential options, why did you select the documented design? (LO\_Explores)\\
  An additional design decision that we considered was incorporating a chatbot tool within our existing system that would help the healthcare professional to pull up a summary of patient's information. It's beneficial as it will ease the healthcare professional's work and save time. However, due to time constraints and complexity, this design decision was dropped.  

\end{enumerate}


\end{document}