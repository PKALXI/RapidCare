\documentclass[12pt, titlepage]{article}

\usepackage{amsmath, mathtools}

\usepackage[round]{natbib}
\usepackage{amsfonts}
\usepackage{amssymb}
\usepackage{graphicx}
\usepackage{colortbl}
\usepackage{xr}
\usepackage{hyperref}
\usepackage{longtable}
\usepackage{xfrac}
\usepackage{tabularx}
\usepackage{float}
\usepackage{siunitx}
\usepackage{booktabs}
\usepackage{multirow}
\usepackage[section]{placeins}
\usepackage{caption}
\usepackage{fullpage}

\hypersetup{
bookmarks=true,     % show bookmarks bar?
colorlinks=true,       % false: boxed links; true: colored links
linkcolor=red,          % color of internal links (change box color with linkbordercolor)
citecolor=blue,      % color of links to bibliography
filecolor=magenta,  % color of file links
urlcolor=cyan          % color of external links
}

\usepackage{array}

\externaldocument{../../SRS/SRS}

\input{../../Comments}
\input{../../Common}

\newcommand{\projname}{RapidCare}

\begin{document}

\title{Module Interface Specification for \progname{}}

\author{\authname}

\date{\today}

\maketitle

\pagenumbering{roman}

\section{Revision History}

\begin{tabularx}{\textwidth}{p{3cm}p{2cm}X}
\toprule {\bf Date} & {\bf Version} & {\bf Notes}\\
\midrule
Jan 14, 2025 & 1.1 & Initial Document\\
% Date 2 & 1.1 & Notes\\
\bottomrule
\end{tabularx}

~\newpage

\section{Symbols, Abbreviations and Acronyms}

See SRS Documentation at \href{https://github.com/Inreet-Kaur/capstone/blob/main/docs/SRS/SRS.pdf} {SRS document}

\begin{tabularx}{\textwidth}{p{2cm}X}
  \toprule
  {\bf Symbol} & {\bf Description}                                                                                     \\
  \midrule
  MG           & Module Guide                                                                                          \\
  M            & Module                                                                                                \\
  MIS          & Module Interface Specification                                                                        \\
  API          & Application Programming Interface                                                                     \\
  MFA          & Multi-Factor Authentication                                                                           \\
  \bottomrule
\end{tabularx}

\newpage

\tableofcontents

\newpage

\pagenumbering{arabic}

\section{Introduction}

The following document details the Module Interface Specifications for the \projname application.

Complementary documents include the System Requirement Specifications and Module Guide. The full documentation and implementation can be found at \url{https://github.com/PKALXI/RapidCare/blob/main/docs/Design/SoftArchitecture/MG.pdf}.

\section{Notation}

% \wss{You should describe your notation.  You can use what is below as
%   a starting point.}

The structure of the MIS for modules comes from \citet{HoffmanAndStrooper1995},
with the addition that template modules have been adapted from
\cite{GhezziEtAl2003}.  The mathematical notation comes from Chapter 3 of
\citet{HoffmanAndStrooper1995}.  For instance, the symbol := is used for a
multiple assignment statement and conditional rules follow the form $(c_1
\Rightarrow r_1 | c_2 \Rightarrow r_2 | ... | c_n \Rightarrow r_n )$.

The following table summarizes the primitive data types used by \progname. 

\begin{center}
\renewcommand{\arraystretch}{1.2}
\noindent 
\begin{tabular}{l l p{7.5cm}} 
\toprule 
\textbf{Data Type} & \textbf{Notation} & \textbf{Description}\\ 
\midrule
character & char & a single symbol or digit\\
integer & $\mathbb{Z}$ & a number without a fractional component in (-$\infty$, $\infty$) \\
natural number & $\mathbb{N}$ & a number without a fractional component in [1, $\infty$) \\
real & $\mathbb{R}$ & any number in (-$\infty$, $\infty$)\\
\bottomrule
\end{tabular} 
\end{center}

\noindent
The specification of \progname \ uses some derived data types: sequences, strings, and
tuples. Sequences are lists filled with elements of the same data type. Strings
are sequences of characters. Tuples contain a list of values, potentially of
different types. In addition, \progname \ uses functions, which
are defined by the data types of their inputs and outputs. Local functions are
described by giving their type signature followed by their specification.

\section{Module Decomposition}

The following table is taken directly from the Module Guide document for this project.

\begin{table}[h!]
\centering
\begin{tabular}{p{0.3\textwidth} p{0.6\textwidth}}
\toprule
\textbf{Level 1} & \textbf{Level 2}\\
\midrule
{Hardware-Hiding} & None \\
\midrule
\multirow{7}{0.3\textwidth}{Behaviour-Hiding} & User Authentication Module\\
& Administrator View Module\\
& Patient View Module\\
& Administrator Model Module\\
& Patient Model Module\\
& Broker Module\\
& Administrator Account Management Module\\
& Patient Account Management Module\\
\midrule
\multirow{3}{0.3\textwidth}{Software Decision} & Transcription Module\\
& Classification Module\\
& Diagnosis Prediction Module\\
& Medicine Prediction Module\\
\bottomrule
\end{tabular}
\caption{Module Hierarchy}
\label{TblMH}
\end{table}

\newpage
~\newpage

\section{MIS of User Authentication Module} \label{Module_UserAuth}

\subsection{Module}
UserAuthentication

\subsection{Uses}
Firebase Auth\\

\subsection{Syntax}

\subsubsection{Exported Constants}
isAuthenticated: Boolean

\subsubsection{Exported Access Programs}

\begin{center}
\begin{tabular}{p{3cm} p{4cm} p{3cm} p{3.5cm}}
\hline
\textbf{Name} & \textbf{In} & \textbf{Out} & \textbf{Exceptions} \\
\hline
Auth & - & React.component &  RenderError\\
\hline
\end{tabular}
\end{center}

\subsection{Semantics}

\subsubsection{State Variables}
isUserAdmin: Boolean

\subsubsection{Environment Variables}
N/A

\subsubsection{Assumptions}
N/A

\subsubsection{Access Routine Semantics}

\noindent Auth():
\begin{itemize}
\item transition: Renders the login page on the screen.
\item output: N/A
\item exception: RenderError — Thrown if the component fails to render.
\end{itemize}

\subsubsection{Local Functions}

\noindent login():
\begin{itemize}
\item transition: Renders adminstrator view page if user is administrator otherwise renders patient view page. 
\item output: N/A
\item exception: InvalidCredentials - Thrown if the user enters invalid credentials.
\end{itemize}

\noindent ResetPassowrd():
\begin{itemize}
\item transition: Sends a reset link to the provided email and renders the login page.
\item output: N/A
\item exception: InvalidInpuError - Thrown if user input a invalid or unregistered email.
\end{itemize}

\newpage


\section{MIS of Administrator View Module} \label{Module_ AdminView}

\subsection{Module}
Administrator

\subsection{Uses}
Broker Module \\
ReactJS\\

\subsection{Syntax}

\subsubsection{Exported Constants}
N/A

\subsubsection{Exported Access Programs}

\begin{center}
\begin{tabular}{p{3cm} p{2cm} p{4cm} p{3cm}}
\hline
\textbf{Name} & \textbf{In} & \textbf{Out} & \textbf{Exceptions} \\
\hline
Administrator & - & React.component &  RenderError\\
\hline
\end{tabular}
\end{center}

\subsection{Semantics}

\subsubsection{State Variables}
isAuthenticated: Boolean

\subsubsection{Environment Variables}
Screen interface\\
Keyboard\\
Microphone

\subsubsection{Assumptions}
User has a functional screen, keyboard, and Microphone.

\subsubsection{Access Routine Semantics}

\noindent Administrator():
\begin{itemize}
\item transition: Renders a react component of the administrator view page.
\item output: N/A
\item exception: RenderError — Thrown if the component fails to render.
\end{itemize}

\subsubsection{Local Functions}

\noindent handleAdminAccount(id: string, record: FormData, requestType: string):
\begin{itemize}
\item transition: Sends an API request to the broker Module to process an add, delete, or update operation in the Administrator Database.
\item output: N/A
\item exception: InvalidInputError - Thrown if the formData is missing a field or is invalid.
\end{itemize} 

\noindent validateInput(InputField: string):
\begin{itemize}
\item transition: Renders a success or error message outlining the action performed.
\item output: N/A
\item exception: InvalidInputError - The input data is incomplete or invalid.
\end{itemize}

\newpage

\section{MIS of Patient View Module} \label{Module_PatientView}

\subsection{Module}
Patient

\subsection{Uses}
Broker Module \\
ReactJS\\

\subsection{Syntax}

\subsubsection{Exported Constants}
N/A

\subsubsection{Exported Access Programs}

\begin{center}
  \begin{tabular}{p{3cm} p{2cm} p{4cm} p{3cm}}
\hline
\textbf{Name} & \textbf{In} & \textbf{Out} & \textbf{Exceptions} \\
\hline
Patient & - & React.component & RenderError \\
\hline
\end{tabular}
\end{center}

\subsection{Semantics}

\subsubsection{State Variables}
isAuthenticated: Boolean

\subsubsection{Environment Variables}
Screen interface\\
Keyboard\\
Microphone\\

\subsubsection{Assumptions}
User has a functional screen, keyboard, and microphone.

\subsubsection{Access Routine Semantics}

\noindent Patient():
\begin{itemize}
\item transition: Renders a react component of the patient view page.
\item output: N/A
\item exception: RenderError — Thrown if the component fails to render.
\end{itemize}

\subsubsection{Local Functions}

\noindent handlePatientAccount(id: string, record: FormData, requestType: string):
\begin{itemize}
\item transition: Sends an API request to the API Module to process an add, delete, or update operation in the Patient Database.
\item output: N/A
\item exception: InvalidInputError - Thrown if the formData is missing a field or is invalid.
\end{itemize}

\noindent validateInput(InputField: string):
\begin{itemize}
\item transition: Renders a success/error message outlining the action performed.
\item output: N/A
\item exception: InvalidInputError - The input data is incomplete or invalid.
\end{itemize}

\newpage

\section{MIS of Broker Module } \label{Module_Broker} Pranav

\subsection{Module}
Broker

\subsection{Uses}

This module facilitates secure authentication and authorization processes. Implements the OAuth 2.0 protocol to manage user authentication, issue access tokens, and validate token requests for secure resource access.

\subsection{Syntax}

\subsubsection{Exported Constants}
\begin{itemize}
    \item \texttt{TOKEN\_EXPIRY}: Defines the duration of token validity (e.g., 3600 seconds).
    \item \texttt{AUTH\_URL}: URL endpoint for authorization.
\end{itemize}

\subsubsection{Exported Access Programs}

\begin{center}
\begin{tabular}{p{3cm} p{4cm} p{4cm} p{3.5cm}}
\hline
\textbf{Name} & \textbf{In} & \textbf{Out} & \textbf{Exceptions} \\
\hline
\texttt{authorize} & Client ID, Scope & Authorization Code & Invalid credentials \\
\texttt{getToken} & Auth Code, Client Secret & Access Token & Expired/invalid code \\
\texttt{validateToken} & Access Token & Boolean & Expired/invalid token \\
\hline
\end{tabular}
\end{center}

\subsection{Semantics}

\subsubsection{State Variables}

\texttt{activeTokens}: Stores active access tokens and their metadata.

\subsubsection{Environment Variables}

\begin{itemize}
    \item Requires a stable database connection for token storage.
    \item Relies on network connectivity for OAuth communications.
\end{itemize}

\subsubsection{Assumptions}

\begin{itemize}
    \item The external client configurations align with OAuth 2.0 standards.
    \item Tokens are used within their defined expiry period.
\end{itemize}

\subsubsection{Access Routine Semantics}

\noindent \texttt{authorize()}:
\begin{itemize}
    \item \textbf{Transition:} Generates an authorization code upon successful client validation.
    \item \textbf{Output:} Authorization code.
    \item \textbf{Exception:} Invalid client credentials.
\end{itemize}

\noindent \texttt{getToken()}:
\begin{itemize}
    \item \textbf{Transition:} Issues an access token and stores it in \texttt{activeTokens}.
    \item \textbf{Output:} Access token.
    \item \textbf{Exception:} Invalid or expired authorization code.
\end{itemize}

\subsubsection{Local Functions}

\texttt{hashSecret(secret: String) -> String}: Hashes the provided client secret for secure storage.

\newpage

\section{MIS of Administrator Model Module } \label{Admin_Model_Module} Gurleen

\subsection{Module}

AdminModel

\subsection{Uses}

N/A

\subsection{Syntax}

\subsubsection{Exported Constants}

N/A

\subsubsection{Exported Access Programs}

\begin{center}
\begin{tabular}{p{3cm} p{4cm} p{4cm} p{3.5cm}}
\hline
\textbf{Name} & \textbf{In} & \textbf{Out} & \textbf{Exceptions} \\
\hline
AdminModel & - & - & - & \\ 
\hline
\end{tabular}
\end{center}

\subsection{Semantics}

\subsubsection{State Variables}

professional_name: string
professional_age: integer
Location: string
Profession: string

\subsubsection{Environment Variables}

N/A

\subsubsection{Assumptions}

N/A

\subsubsection{Access Routine Semantics}

\noindent \texttt{getter()}: This is the boiler state of this variable where it will get a certain start and return it.
\begin{itemize}
    \item \textbf{Transition:} N/A
    \item \textbf{Output:} The output depends on the data type of the parameter. It returns the current value of the requested data element.
    \item \textbf{Exception:} N/A
\end{itemize}

\noindent \texttt{setter()}: This is the boiler state of this variable where it will get a certain start and return it.
\begin{itemize}
    \item \textbf{Transition:} Updates the internal state of the data model, either by adding or updating data. This also changes certain state variables.
    \item \textbf{Output:} N/A
    \item \textbf{Exception:} N/A
\end{itemize}

\subsubsection{Local Functions}

\noindent init(inputField: string):
\begin{itemize}
\item transition: N/A
\item output: Necessary data structures and connections are made to manage and access data.
\item exception: N/A
\end{itemize}

\newpage

\section{MIS of Patient Model Module } \label{Patient_Model_Module} Gurleen

\subsection{Module}

PatientModule

\subsection{Uses}

N/A

\subsection{Syntax}

\subsubsection{Exported Constants}

N/A

\subsubsection{Exported Access Programs}

\begin{center}
\begin{tabular}{p{3cm} p{4cm} p{4cm} p{3.5cm}}
\hline
\textbf{Name} & \textbf{In} & \textbf{Out} & \textbf{Exceptions} \\
\hline
AdminModel & - & - & - & \\ 
\hline
\end{tabular}
\end{center}

\subsection{Semantics}

\subsubsection{State Variables}

patient_name = string
date_of_birth = date
patient_age = integer
gender = string
home_address = string
email_address = string
contact_number = string
past_medical_history = string
allergies(if any) = string
family_history = string
current_medications = string
medical_record_number = string
insurance_number = string

\subsubsection{Environment Variables}

N/A

\subsubsection{Assumptions}

N/A

\subsubsection{Access Routine Semantics}

\noindent \texttt{getter()}: This is the boiler state of this variable where it will get a certain start and return it.
\begin{itemize}
    \item \textbf{Transition:} N/A
    \item \textbf{Output:} The output depends on the data type of the parameter. It returns the current value of the requested data element.
    \item \textbf{Exception:} N/A
\end{itemize}

\noindent \texttt{setter()}: This is the boiler state of this variable where it will get a certain start and return it.
\begin{itemize}
    \item \textbf{Transition:} Updates the internal state of the data model, either by adding or updating data. This also changes certain state variables
    \item \textbf{Output:} N/A
    \item \textbf{Exception:} N/A
\end{itemize}

\subsubsection{Local Functions}

\noindent init(inputField: string):
\begin{itemize}
\item transition: N/A
\item output: Necessary data structures and connections are made to manage and access data.
\item exception: N/A
\end{itemize}

\newpage

  
\section{MIS of Transcription Module} \label{Transcription_Module}

\subsection{Module}
TranscriptionModule

\subsection{Uses}

N/A

\subsection{Syntax}

\subsubsection{Exported Constants}

N/A

\subsubsection{Exported Access Programs}

\begin{center}
\begin{tabular}{p{2cm} p{4cm} p{4cm} p{2cm}}
\hline
\textbf{Name} & \textbf{In} & \textbf{Out} & \textbf{Exceptions} \\
\hline
Transcription & Audio: bytes & TranscribedText: string & InvalidInputError \\ 
\hline
\end{tabular}
\end{center}

\subsection{Semantics}

\subsubsection{State Variables}

N/A

\subsubsection{Environment Variables}

\begin{itemize}
  \item Microphone: It needs to access the microphone so that the conversation between the patient and healthcare professional is recorded.
\end{itemize}

\subsubsection{Assumptions}

\begin{itemize}
  \item It is assumed that the microphone works correctly in the system.
\end{itemize}

\subsubsection{Access Routine Semantics}

\noindent TranscriptionModule(AudioData: bytes):
\begin{itemize}
  \item transition: N/A
  \item output: Transcribed text transcribed from the audio bytes. 
  \item exception: InvalidInputError - If the bytes could not be converted to text. 
\end{itemize}

\subsubsection{Local Functions}

N/A

\newpage
~\newpage
  
\section{MIS of Classification Module} \label{Classification_Module} 

\subsection{Module}

ClassificationModule

\subsection{Uses}

N/A

\subsection{Syntax}

\subsubsection{Exported Constants}

N/A

\subsubsection{Exported Access Programs}

\begin{center}
\begin{tabular}{p{2cm} p{4cm} p{4cm} p{2cm}}
\hline
\textbf{Name} & \textbf{In} & \textbf{Out} & \textbf{Exceptions} \\
\hline
Classification & TranscribedText: string & ClassifiedText: string & - & \\
\hline
\end{tabular}
\end{center}

\subsection{Semantics}

\subsubsection{State Variables}

N/A

\subsubsection{Environment Variables}

N/A

\subsubsection{Assumptions}

\begin{itemize}
  \item It is assumed that the transcribed text is in English language.
\end{itemize}

\subsubsection{Access Routine Semantics}

\noindent ClassifyModule(TranscribedText: string):
\begin{itemize}
\item transition: N/A
\item output: Classified text generated for report generation. 
\item exception: N/A 
\end{itemize}

\subsubsection{Local Functions}

N/A

\newpage

\section{MIS of Diagnosis Prediction Module} \label{diag_pred_mod}

\subsection{Module}

DiagnosisPred

\subsection{Uses}
\begin{itemize}
  \item Report Generation Module. Specific inputs for exported access programs:
    \begin{itemize}
      \item Past medical history : text
      \item Symptoms : text
      \item User characteristics :
        \begin{itemize}
          \item Age : integer
          \item Weight : integer
        \end{itemize}
    \end{itemize}
  \item Preprocessing Module
  \item Tensorflow
  \item Scikit-Learn
  \item Flask
  \item Flask-CORS
  \item Diagnostic Database (for training the model) [\ref{diag_data}]
\end{itemize}

\subsection{Syntax}

\subsubsection{Exported Constants}
The exported constants for this module would a prediction of the possible diagnosis.

\subsubsection{Exported Access Programs}

\begin{center}
\begin{tabular}{p{2cm} p{4cm} p{4cm} p{2cm}}
\hline
\textbf{Name} & \textbf{In} & \textbf{Out} & \textbf{Exceptions} \\
\hline
Flask Application (Python) & Symptoms, Past medical History, User characteristics & Prediction of possible Diagnosis & InputDimError \\
\hline
\end{tabular}
\end{center}

\subsection{Semantics}

\subsubsection{State Variables}

\begin{itemize}
  \item model: Tensorflow model
  % \item modelPipline : Abstract Object
\end{itemize}

\subsubsection{Environment Variables}

There will be no environment variables that this module will interact with.

\subsubsection{Assumptions}

\begin{itemize}
  \item Patients are not making up symptoms and all input features are accurate.
  % \item Training pipeline for the model is setup
\end{itemize}

\subsubsection{Access Routine Semantics}

\noindent diagnosePatient(request : FormData):
\begin{itemize}
\item transition: Preprocess the request body (FormData) which contain the inputs as specified above using the preprocess local function. 
\item output: Returns the predicted diagnosis for the patient.
\item exception: InputDimError - The expected request body items were not received or were in the wrong format.
\end{itemize}

\subsubsection{Local Functions}

\noindent preProcessData(pastHistory: String, symptoms: String, user -> \{age, weight\}):
\begin{itemize}
\item transition: Preprocess the text using TF-ID and normalize the continuous inputs.
\item output: Return the preprocessed data.
\item exception: InputDimError - The expected arguments were not received or were in the wrong format.
\end{itemize}

\newpage
  
\section{MIS of Medicine Prediction Module} \label{med_pred_mod}


\subsection{Module}

MedPred

\subsection{Uses}
\begin{itemize}
  \item Report Generation Module. Specific inputs for exported access programs:
    \begin{itemize}
      \item Past medical history : text
      \item Symptoms : text
      \item User characteristics :
        \begin{itemize}
          \item Age : integer
          \item Weight : integer
        \end{itemize}
      \item Physician confirmed diagnosis.
    \end{itemize}
  \item Preprocessing Module
  \item Tensorflow
  \item Flask
  \item Flask-CORS
  \item Scikit-Learn
  \item Medicine Database (for training the model) [\ref{med_data}]

\end{itemize}

\subsection{Syntax}

\subsubsection{Exported Constants}
The exported constants for this module would be a prediction of the possible medicine to treat the diagnosis.

\subsubsection{Exported Access Programs}

\begin{center}
\begin{tabular}{p{2cm} p{4cm} p{4cm} p{2cm}}
\hline
\textbf{Name} & \textbf{In} & \textbf{Out} & \textbf{Exceptions} \\
\hline
Flask Application (Python) & Symptoms, Past medical History, User characteristics, Diagnosis & Prediction of possible Medicine & InputDimError \\
\hline
\end{tabular}
\end{center}

\subsection{Semantics}

\subsubsection{State Variables}

\begin{itemize}
  \item model: Tensorflow model
  % \item modelPipeline : Abstract Object
\end{itemize}

\subsubsection{Environment Variables}

There will be no environment variables that this module will interact with.

\subsubsection{Assumptions}


\begin{itemize}
  \item All input features are accurate.
  % \item Training pipeline for the model is setup
\end{itemize}

\subsubsection{Access Routine Semantics}

\noindent medicatePatient(request : FormData):
\begin{itemize}
\item transition: preprocess the request body (FormData) which contain the inputs as specified above using the preprocess local function. 
\item output: Returns the predicted medicine for the patient.
\item exception: InputDimError - The expected request body items were not received or were in the wrong format.
\end{itemize}

\subsubsection{Local Functions}

\noindent preProcessData(diagnosis: String, pastHistory: String, symptoms: String, user -> \{age, weight\}):
\begin{itemize}
\item transition: Preprocess the text using TF-ID and normalize the continuous inputs, finally the diagnosis will be encoded using a LabelEncoder.
\item output: Return the preprocessed data.
\item exception: InputDimError - The expected arguments were not received or were in the wrong format.
\end{itemize}

\newpage

\section{MIS of Administrator Account Management Module} \label{Module_AdminAccount}

\subsection{Module}
AdministratorAccountManagement

\subsection{Uses}
Broker Module\\

\subsection{Syntax}

\subsubsection{Exported Constants}
N/A

\subsubsection{Exported Access Programs}

\begin{center}
\begin{tabular}{p{4cm} p{4.5cm} p{1cm} p{2.5cm}}
\hline
\textbf{Name} & \textbf{In} & \textbf{Out} & \textbf{Exceptions} \\
\hline
addHealthcareProfessional & id:string, value: FormData & - & InvalidInputError \\
deleteHealthcareProfessional & id:string & - & IdNotFound\\
updateHealthcareProfessional & id:string, value: FormData & - & InvalidInputError \\
\hline
\end{tabular}
\end{center}

\subsection{Semantics}

\subsubsection{State Variables}
dbConnection : Database connection point.

\subsubsection{Environment Variables}
N/A

\subsubsection{Assumptions}
Database containing the account information for healthcare professionals exists.

\subsubsection{Access Routine Semantics}

\noindent addHealthcareProfessional(id:string, record: FormData):
\begin{itemize}
\item transition: Adds a new document with the provided details to the database.
\item output: N/A
\item exception: InvalidInputError - The input data is incomplete or invalid or a duplicate document already exists.
\end{itemize}

\noindent deleteHealthcareProfessional(id:string):
\begin{itemize}
\item transition: Deletes the corresponding document from the database.
\item output: N/A
\item exception: IdNotFound - Provided id is invalid or does not exist.
\end{itemize}

\noindent updateHealthcareProfessional(id:string, record: FormData):
\begin{itemize}
\item transition: Update document with the provided details in the database.
\item output: N/A
\item exception: InvalidInputError - The input data is incomplete or invalid.
\end{itemize}

\subsubsection{Local Functions}

N/A

\newpage

\section{MIS of Patient Account Management Module} \label{Module_PatientAccountManag}

\subsection{Module}
PatientAccountManagement

\subsection{Uses}
Broker Module\\

\subsection{Syntax}

\subsubsection{Exported Constants}
N/A

\subsubsection{Exported Access Programs}

\begin{center}
\begin{tabular}{p{4cm} p{4.5cm} p{1cm} p{2.5cm}}
\hline
\textbf{Name} & \textbf{In} & \textbf{Out} & \textbf{Exceptions} \\
\hline
createPatientRecord & id:string, value: FormData & - & InvalidInputError \\
deletePatientRecord & id:string & - & IdNotFound\\
updatePatientRecord & id:string, value: FormData & - & InvalidInputError \\
\hline
\end{tabular}
\end{center}

\subsection{Semantics}

\subsubsection{State Variables}
dbConnection : Database connection point.

\subsubsection{Environment Variables}
N/A

\subsubsection{Assumptions}
Database containing the account information for the patients exists.

\subsubsection{Access Routine Semantics}

\noindent createPatientRecord(id:string, record: FormData):
\begin{itemize}
\item transition: Adds a new document with the provided details to the database.
\item output: N/A
\item exception: InvalidInputError - The input data is incomplete or invalid or a duplicate document already exists.
\end{itemize}

\noindent deletePatientRecord(id:string):
\begin{itemize}
\item transition: Deletes the corresponding document from the database.
\item output: N/A
\item exception: IdNotFound - Provided id is invalid or does not exist.
\end{itemize}

\noindent updatePatientRecord(id:string, record: FormData):
\begin{itemize}
\item transition: Update document with the provided details in the database.
\item output: N/A
\item exception: InvalidInputError - The input data is incomplete or invalid.
\end{itemize}

\subsubsection{Local Functions}

N/A

\newpage


\bibliographystyle {plainnat}
\bibliography {../../../refs/References}

\newpage

\section{Appendix} \label{Appendix}

\wss{Extra information if required}

\newpage{}

\section*{Appendix --- Reflection}

\wss{Not required for CAS 741 projects}

The information in this section will be used to evaluate the team members on the
graduate attribute of Problem Analysis and Design.

\input{../../Reflection.tex}

\begin{enumerate}
  \item What went well while writing this deliverable? 
  \item What pain points did you experience during this deliverable, and how
    did you resolve them?
  \item Which of your design decisions stemmed from speaking to your client(s)
  or a proxy (e.g. your peers, stakeholders, potential users)? For those that
  were not, why, and where did they come from?
  \item While creating the design doc, what parts of your other documents (e.g.
  requirements, hazard analysis, etc), it any, needed to be changed, and why?
  \item What are the limitations of your solution?  Put another way, given
  unlimited resources, what could you do to make the project better? (LO\_ProbSolutions)
  \item Give a brief overview of other design solutions you considered.  What
  are the benefits and tradeoffs of those other designs compared with the chosen
  design?  From all the potential options, why did you select the documented design?
  (LO\_Explores)
\end{enumerate}


\end{document}