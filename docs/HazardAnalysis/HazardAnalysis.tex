\documentclass{article}

\usepackage{booktabs}
\usepackage{tabularx}
\usepackage{hyperref}
\usepackage{ulem}

\usepackage{amsmath, mathtools}
\usepackage{amsfonts}
\usepackage{amssymb}
\usepackage{graphicx}
\usepackage{colortbl}
\usepackage{xr}
\usepackage{longtable}
\usepackage{xfrac}
\usepackage{float}
\usepackage{siunitx}
\usepackage{caption}
\usepackage{pdflscape}
\usepackage{afterpage}
\usepackage{float}
\usepackage{array}
\usepackage[round]{natbib}

% For easy change of table widths
\newcommand{\colZwidth}{1.0\textwidth}
\newcommand{\colAwidth}{0.13\textwidth}
\newcommand{\colBwidth}{0.82\textwidth}
\newcommand{\colCwidth}{0.1\textwidth}
\newcommand{\colDwidth}{0.05\textwidth}
\newcommand{\colEwidth}{0.8\textwidth}
\newcommand{\colFwidth}{0.17\textwidth}
\newcommand{\colGwidth}{0.5\textwidth}
\newcommand{\colHwidth}{0.28\textwidth}

\newcounter{irnum} %IR Number
\newcommand{\rtheirnum}{IR\theirnum}
\newcommand{\irref}[1]{IR\ref{#1}}

\newcounter{acnum} %AC Number
\newcommand{\rtheacnum}{AC\theacnum}
\newcommand{\acref}[1]{AC\ref{#1}}

\newcounter{prnum} %PR Number
\newcommand{\rtheprnum}{PR\theprnum}
\newcommand{\prref}[1]{PR\ref{#1}}

\newcounter{adnum} %AD Number
\newcommand{\rtheadnum}{AD\theadnum}
\newcommand{\adref}[1]{AD\ref{#1}}

\newcounter{imnum} %IM Number
\newcommand{\rtheimnum}{IM\theimnum}
\newcommand{\imref}[1]{IM\ref{#1}}


\hypersetup{
    colorlinks=true,       % false: boxed links; true: colored links
    linkcolor=red,          % color of internal links (change box color with linkbordercolor)
    citecolor=green,        % color of links to bibliography
    filecolor=magenta,      % color of file links
    urlcolor=cyan           % color of external links
}

\title{Hazard Analysis\\\progname}

\author{\authname}

\date{}

\input{../Comments}
\input{../Common}

\begin{document}

\maketitle
\thispagestyle{empty}

~\newpage

\pagenumbering{roman}

\begin{table}[hp]
\caption{Revision History} \label{TblRevisionHistory}
\begin{tabularx}{\textwidth}{llX}
\toprule
\textbf{Date} & \textbf{Developer(s)} & \textbf{Change}\\
\midrule
18-10-2024 & Inreet Kaur & FMEA Table, Critical Assumptions\\
18-10-2024 & Moamen Ahmed & FMEA Table, Safety + Security Reqs.\\
20-10-2024 & Gurleen Rahi & FMEA Table, Safety + Security Reqs.\\
20-10-2024 & Inreet Kaur & System Composition\\
24-10-2024 & Inreet Kaur & Intro, Scope\\
24-10-2024 & Pranav Kalsi & FMEA Table, Safety + Security Reqs, Roadmap, Document Compilation Fix..\\
25-10-2024 & Gurleen Rahi & Reflection.\\
\bottomrule
\end{tabularx}
\end{table}

~\newpage

\tableofcontents
% \listoffigures 

~\newpage

\pagenumbering{arabic}

% \wss{You are free to modify this template.}

\section{Introduction}

The purpose of this document is to provide a comprehensive hazard analysis for RapidCare, a software application that aims to streamline the healthcare documentation process. According to Nancy Leveson, hazard is a property/condition within the system and its environment that can cause harm or result in loss [1]. To ensure the safety of the system as well as the user, it is critical to identify and mitigate potential hazards.

For the purposes of this document, we will use the Failure Modes and Effect Analysis (FMEA) method for hazard analysis. This document will provide an overview of the scope and purpose of hazard analysis, system boundaries and components, critical assumptions about the system and its environment, and an FMEA table listing the causes and effects of failure along with recommended actions. The document will also list any additional safety and security requirements identified as a result of hazard analysis and a roadmap for implementation.

\section{Scope and Purpose of Hazard Analysis}

Hazards can arise from various sources such as user input, security issues, system failure or other external factors where the system is deployed. The scope of this document is a hazard within the various system components as well as the environment in which the system will operate.

The purpose of hazard analysis is to proactively identify all potential hazards, the effects and causes of the failure and to develop appropriate mitigation strategies. Since the system will operate in a healthcare setting, it is critical to identify potential hazards. This will ensure the safety, reliability, and security of the system. Moreover, it is essential to protect sensitive information, delays in treatment, other medical errors, and the safety of the system and user. 

\section{System Boundaries and Components}

To identify potential hazards, we first define the system boundaries and break it down into its major components:

\begin{itemize}
 
    \item \textbf{User Interface:}
    The user interface is the point of interaction between the users and the system. It is responsible for displaying outputs from the system, such as patient data, medication suggestions, diagnosis predictions etc. The UI plays a crucial role in ensuring a user-friendly and intuitive experience for the users.
    
    Potential Hazards:
    \begin{itemize}
        \item Incorrect user input
        \item Inadequate feedback when errors occur
        \item Incorrect data displayed to the user
    \end{itemize}

    \item \textbf{Data Layer:}
    The data layer in the system is responsible for managing and processing all data related to patient records, healthcare professionals, health networks, and predictive models for medication and diagnosis. It is divided into the following databases:
    \begin{itemize}
        \item DB1: Patient, Healthcare Professional, and Network Database: This database stores patient records, healthcare professionals, and healthcare network profiles. This component is responsible for storing, retrieving, updating, and deleting data. 

        \item DB2: Diagnosis Prediction Database: This database stores the data used by the diagnosis prediction component to suggest potential diagnoses based on analysis of the transcribed data.

        \item DB3: Medication Prediction Database: This database holds the data used by the medical prediction component to suggest appropriate medications based on the identified or accepted diagnosis.
    \end{itemize}

    Potential Hazards:
    \begin{itemize}
        \item Accidental deletion of database entries or the entire database
        \item Creation of duplicate records
        \item Security breaches
        \item Database crashes   
    \end{itemize}
    
    \item \textbf{User Authentication:}
    This component verifies and validates user credentials for secure system access. It implements multi-factor authentication and integrates with the system's cryptographic infrastructure. The component also handles password security and implements measures against unauthorized access attempts.
    
    Potential Hazards:
    \begin{itemize}
        \item User cannot log in to the system
    \end{itemize}
    
    \item \textbf{Account Management:}
    This component oversees user account lifecycles within the system. It handles account creation, profile updates, and account deletion, ensuring data integrity throughout these processes.
    
    Potential Hazards:
    \begin{itemize}
        \item Account cannot be created, updated, or deleted
    \end{itemize}
    
    \item \textbf{Transcription Module:}
    The transcription module is responsible for converting audio data from the conversation to text. The converted written text is used thereafter used by the report generation module to generate the report of the patient.

    Potential Hazards:
    \begin{itemize}
        \item Incorrect transcription from audio to text
    \end{itemize}

    \item \textbf{Classification Module:}
    The classification module is responsible for extracting the key information from the transcription (such as symptoms, reason for visit, allergies, current medications, etc..) and automatically fill out the patient health document.

    Potential Hazards:
    \begin{itemize}
        \item Incorrect classification from text to document
    \end{itemize}

    \item \textbf{Prediction Modules:}
    This module is responsible for all of the predictions that will occur based on the medical notes.

    \begin{itemize}
        \item PR1: Diagnosis Prediction: The prediction module is responsible for using the clinical notes to provide some diagnosis predictions. The goal of this module is to associate symptoms with common diagnoses. This allows for a lateral view of possible patient diagnoses.
        \item PR2: Medicine Prediction: Once a diagnosis is selected through the model or manual entry, the corresponding frequently used medication will also be provided.
        \item PR3: AI assist: If the healthcare professional wants to review patient previous visits and health conditions, it can be done by running a query in AI assist which will fetch the requested information from the system.  
    \end{itemize}
    
    Potential Hazards:
    \begin{itemize}
        \item Wrong diagnosis family is predicted
        \item Processing non-medical related inputs
    \end{itemize}
    
\end{itemize}


\section{Critical Assumptions}

The following assumptions are made regarding both the software and hardware components of the system:

\begin{itemize}
    \item \textbf{Stable Network Connection:} It is assumed that the network connection between the client and server will be stable. If the connection is unstable, it could cause interruptions to the process, which results in significant issues in the system’s performance.
        
    \item \textbf{Reliable Hardware:} It is assumed that there won’t be any major hardware failures. Although hardware problems are rare, they could severely affect system availability and accuracy, especially in critical healthcare environment.

    \item \textbf{Well Intentioned Inputs:} We will assume that the user will enter input with good intention and will not be attempting to break the system or overload the system with the requests.
    
    \item \textbf{User Expertise:} We will assume that users possess the domain expertise, such as medical terminologies, which is required to effecitvely interact with the system and validating the report generated by the software.
\end{itemize}

\begin{landscape} 

\section{Failure Mode and Effect Analysis}

The following FMEA table lists all potential hazards related to identified system components above along with the failure mode, effects and causes of failure, detection, and recommended actions. It also provides tracibility to associated requirements.\\
    
    \begin{longtable}{|p{1.5cm}|p{2cm}|p{2.6cm}|p{2cm}|p{2cm}|p{2cm}|p{3.5cm}|p{1cm}|p{0.8cm}|}
        \toprule
        \textbf{Comp.} & \textbf{Design Function} & \textbf{Failure mode} & \textbf{Effects of failure} & \textbf{Causes of failure} & \textbf{Detection} & \textbf{Recommended action} & \textbf{Req.} & \textbf{Ref.}\\ 
        \midrule
        User Interface & \raggedright Allow user to provide data through input for processing & \raggedright System accepts incorrect data & \raggedright Incorrect data stored in the database; Incorrect diagnosis; Inaccurate data may lead to medical errors & \raggedright Lack of input validation; Insufficient feedback mechanism & \raggedright User reports; Record validation checks & \raggedright Improve UI design for discoverability and use appropriate signifiers for various data fields. Display soft feedback to guide user input i.e. implement input masks, field-level validation, and page-level validation to prevent the system from saving any invalid data. Implement constraints on input data fields. & NFR1; NFR2; IR\ref{IR_ErrorDetection} & H1.1 \\ 
        \midrule
        & \raggedright Display error messages and provide feedback & \raggedright System provides inadequate feedback when errors occur & \raggedright Users are unaware of the current system state; Unresolved issues; Inaccurate data stored in a database & \raggedright Insufficient feedback mechanism & \raggedright Error logs; User reports; Record validation checks & \raggedright Provide clear and actionable error messages when an error occurs. Use language familiar to the user for easy interpretation. Provide steps to recover from the error state. & NFR1; NFR2; IR\ref{IR_ErrorDetection} & H1.2 \\
        \midrule
        & \raggedright Display correct data to the user & \raggedright System displays incorrect data to the user & \raggedright Incorrect medical decisions; Compromise patient safety & \raggedright Data processing errors; System bugs & \raggedright User reports; Error logs; Record validation checks & \raggedright Ensure user input is accurately interpreted and stored by the system. Add data verification steps to ensure the system retrieves the correct data to display. & IR\ref{IR_DuplicateRecordDetection} & H1.3 \\
        \midrule
        Data Layer & \raggedright Manage and store data in a secure manner & \raggedright Accidental deletion of database entries or the entire database & \raggedright Permanent loss of critical data & \raggedright User errors; Lack of validation checks & \raggedright User reports; Failure to retrieve or access a data instance or database & \raggedright Display appropriate feedback before confirming the deletion. Implement role-based access control for deletion action. & FR5; FR9; FR2 & H2.1 \\ 
        \midrule
        & & \raggedright Creation of duplicate records & \raggedright Incorrect output displayed to the user; Medical errors & \raggedright Lack of validation on user input & \raggedright Record validation checks & \raggedright Implement validation checks for user input. Implement validation checks before storing a new entry. Regular data integrity checks. & IR\ref{IR_DuplicateRecordDetection} & H2.2 \\ 
        \midrule
        & & \raggedright Security breaches & \raggedright Unauthorized access to sensitive data; Regulatory and compliance issues & \raggedright Improper authentication & \raggedright Security audits; Access logs & \raggedright Implement strong authentication protocols. Ensure compliance with PIPEDA and regulatory standards. & NFR6; FR7 & H2.3 \\ 
        \midrule
        & \raggedright Retrieve and store data in real-time. & \raggedright Database crashes & \raggedright Inability to access stored data; Inability to store new data & \raggedright Server overload; System failure & \raggedright Error messages; Monitoring system performance & \raggedright Implement failover systems. Implement scalable server infrastructure. & NFR4; NFR5 & H2.4 \\ 
        \midrule
        General & \raggedright Provide continuous access to the system & \raggedright App closes unexpectedly & \raggedright Unsaved progress is lost; Delayed medical access to patients & \raggedright Loss of power or internet; Software failure & \raggedright User reports; System logs & \raggedright Implement automatic data backups and recovery system. & NFR4 & H3 \\
        \midrule
        User Authentication & \raggedright Verify user credentials & \raggedright User cannot log in to the system & \raggedright User cannot access any system data or functions & \raggedright Invalid credentials; Database failure & \raggedright Failed login attempts & \raggedright Reset credentials. Verify database connectivity & IR\ref{IR_Autentication} & H5 \\
        \midrule
        Account Management & \raggedright Manage user accounts (create, update, delete) & \raggedright Account cannot be created, updated, or deleted & \raggedright User unable to register, update info, or remove account & \raggedright Database failure; Validation errors & \raggedright Log account creation, update, and deletion attempts & \raggedright Check database integrity. Implement Validation checks for inputs. & AC\ref{AC_AuthorizedPersonnel} & H6 \\ 
        \midrule
        Transcription Module & \raggedright Convert audio data from the conversation to text & \raggedright System provides incorrect transcription & \raggedright Inaccurate diagnosis; Medical errors & \raggedright Background noise disruption; Misinterpretation of the pronounced words & \raggedright User reports & \raggedright Use models with high transcription accuracy. Prompt user to review the transcribed data. Allow user to validate and edit transcribed data. & FR11; NFR3; IR\ref{IR_BackNoiseFilter} & H8 \\ 
        \midrule
        Classification Module & \raggedright Classify and populates the transcribed text into respective fields & \raggedright System performs the classification incorrectly & \raggedright Incorrect classification from text to document & \raggedright Lack of input validation & \raggedright User reports; Record validation checks & \raggedright Allow user to validate and edit the classified data. Prompt user to review the classified data. & FR11 & H9 \\
        \midrule
        Prediction Module & \raggedright Use medical notes outline standard practices for various procedures to predict a diagnosis along with a treatment plan. & \raggedright Incorrect due to biased prediction & \raggedright Healthcare professional may be misled. & \raggedright Poorly trained model; Biased data. & \raggedright Use validation and cross-validation to evaluate the models. & \raggedright Use healthcare professional evaluation and train systematically. & IR\ref{IR_ValidationScore} & H10.1 \\
        \midrule
        & \raggedright Incorrect diagnosis and treatment plan that have to deviate from standard practice documentation. & \raggedright Processing non-medical related inputs & \raggedright Healthcare professional may be misled. & \raggedright Inputs for model are not appropriately filtered. & \raggedright Add filters to the model pipeline to ensure data inputted is useful data. & \raggedright Add filters to check for quantitative inputs. & - & H10.2 \\ 
        \bottomrule
        \caption{\bf Failure Mode and Effect Analysis of the System}
    \end{longtable}
    
    
\end{landscape}

\section{Safety and Security Requirements}

\subsection{Access Requirements}
\begin{itemize}
    \item [AC\refstepcounter{acnum}\theacnum \label{AC_Authentication}:] The system must allow only authenticated users access to system resources.\\
    \textbf{Rationale:} Authentication is fundamental to system security as it ensures that only verified users can access sensitive resources. Logging failed attempts is crucial for detecting potential security breaches.
    \item [AC\refstepcounter{acnum}\theacnum \label{AC_AuthorizedPersonnel}:] Only authorized personnel can create, update, or delete user accounts.\\
    \textbf{Rationale:} Restricting user management operations to authorized personnel prevents unauthorized account creation and modification, which could lead to security breaches.
\end{itemize}


\subsection{Integrity Requirements}
\begin{itemize}
    \item [IR\refstepcounter{irnum}\theirnum \label{IR_Autentication}:] User credentials must remain intact during authentication. Failed login attempts should not affect the system's functionality or stored data.\\
    \textbf{Rationale:} Maintaining the integrity of user credentials during authentication is essential to prevent unauthorized access and data corruption. The system must remain stable and secure regardless of authentication failures to ensure continuous service availability and protect stored credentials from potential tampering or corruption.

    \item [IR\refstepcounter{irnum}\theirnum \label{IR_ErrorDetection}:] The system should provide real-time error detection based on validation checks and provide feedback to users.\\
    \textbf{Rationale:}To prevent user errors and incorrect output, it is vital to check the integrity of user input. Moreover, in the event of an error, the system should communicate its current state, how the input has been interpreted, and any related errors to the user.

    \item [IR\refstepcounter{irnum}\theirnum \label{IR_ValidationScore}:] The system should provide confidence score on any prediction. Additionally, after each model training session validation accuracy score must be outputted.\\
    \textbf{Rationale:} To understand the bias in any data and in the model. Using validation scores for training and test sets the model can be evaluated to detect underfitting and overfitting. Additionally, adding model confidence scores will allow  the healthcare professionals to see how confident the model is with the scores.
    
    \item [IR\refstepcounter{irnum}\theirnum \label{IR_DuplicateRecordDetection}:] The system should provide duplicate record detection for the record in various databases of the system.\\
    \textbf{Rationale:} To prevent confusion and medical errors resulting from duplicate entries, the system should validate and flag potential duplicate records before they are created.

    \item [IR\refstepcounter{irnum}\theirnum \label{IR_BackNoiseFilter}:] The system should be able to filter the background noise to produce accuracy transcribed data.\\
    \textbf{Rationale:} It is essential to produce accurate data to avoid any inaccurate records, potential medical errors and incorrect diagnosis. 

\end{itemize}

\subsection{Privacy Requirements}
Covered in SRS

\subsection{Audit Requirements}
Covered in SRS

\subsection{Immunity Requirements}
Covered in SRS

\section{Roadmap}

% \wss{Which safety requirements will be implemented as part of the capstone timeline?
% Which requirements will be implemented in the future?}

After the hazard analysis we identified a lot of new safety and security requirements. In terms of the scope of the capstone project, in terms of these requirements the team will aim to deliver all of the mentioned requirements in this document including the mentioned access and integrity requirements. To add in our feedback we will first prioritize critical components and incrementally integrate changes by assigning the tickets to the respective developers.

IR\ref{IR_ValidationScore} may be simplified to just attempting to mitigate model bias. The roadmap will re-assessed towards the end of the project to see if features can be augmented to complete the requirements.


\newpage{}

\section{References}

\begin{itemize}
    \item [1] N. G. Leveson and J. P. Thomas, “STPA handbook (MIT-Stamp-001),” STPA Handbook, https://psas.scripts.mit.edu/home/get_file.php?name=STPA_handbook.pdf (accessed Oct. 9, 2024).
\end{itemize}

\newpage{}

\section*{Appendix --- Reflection}

% \wss{Not required for CAS 741}

\input{../Reflection.tex}

\begin{enumerate}
    \item What went well while writing this deliverable?\\
    This document has let us build more on the critical hazards and mitigation strategies to overcome them. While going through the outline of this document, we were able to provide the system components as well as their failure modes of the system. 
    
    \item What pain points did you experience during this deliverable, and how did you resolve them?\\
    Every team project has challenges that must be solved in order to move forward successfully. We had to create a donation plan to guarantee smooth operations. To ensure that all of our plans are in line with the system, we need to plan the system components that complement our project. In order to contribute to the document and evaluate each other's work as effectively as possible, we also needed to set up a schedule.

    \item Which of your listed risks had your team thought of before this deliverable, and which did you think of while doing this deliverable? For the latter ones (ones you thought of while doing the Hazard Analysis), how did they come about?\\
    Our team had some knowledge of the system's components and possible software hazards prior to this deliverable. However, we also came up with some safety requirements, including integrity requirements, while working on this deliverable. Integrity requirements have listed a lot of detection modules that are essential for authentication purposes.
    
    \item Other than the risk of physical harm (some projects may not have any appreciable risks of this form), list at least 2 other types of risk in software products. Why are they important to consider?\\
    The other two categories of risks are security risks, like cyber-attacks, and functional risks, like performance problems. They are crucial since a malfunctioning system could cause the automation process to be delayed, which would defeat the goal of the project. Additionally, phishing and hacking may result in the loss of patient data, which would affect data privacy. 

\end{enumerate}

\end{document}