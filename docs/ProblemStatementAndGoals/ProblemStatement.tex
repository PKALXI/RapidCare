\documentclass{article}

\usepackage{tabularx}
\usepackage{booktabs}
\usepackage{float}

\title{Problem Statement and Goals\\\progname}

\author{\authname}

\date{September 24, 2024}

\input{../Comments}
\input{../Common}

\begin{document}

\maketitle

\begin{table}[hp]
\caption{Revision History} \label{TblRevisionHistory}
\begin{tabularx}{\textwidth}{llX}
\toprule
\textbf{Date} & \textbf{Developer(s)} & \textbf{Change}\\
\midrule
24--09--2024 & Pranav Kalsi & Problem Statement Excluding 1.2\\
24--09--2024 & Gurleen Rahi & 1.2 - Inputs and Outputs\\
24--09--2024 & Pranav Kalsi & Challenge Level and Extras\\
24--09--2024 & Pranav Kalsi, Moamen Ahmed & Goals + Stretch Goals\\
24--09--2024 & All & Reflection\\
04--01--2025 & Moamen Ahmed & Updated Goals according to TA's feedback\\
16--03--2025 & Gurleen Rahi & Revised Inputs and Outputs + Extras + Goals + Stretch Goals according to the TA's feedback\\ 
\bottomrule
\end{tabularx}
\end{table}

\section{Problem Statement}
% \wss{You should check your problem statement with the
% \href{https://github.com/smiths/capTemplate/blob/main/docs/Checklists/ProbState-Checklist.pdf}
% {problem statement checklist}.} 
% \wss{You can change the section headings, as long as you include the required
% information.}

\subsection{Problem}

Ontario is facing an extreme shortage of family doctors, with the number of patients without one jumping by 600,000 to 2.5 million which is a growing number [1]. This situation is only to get worse as predicted by the Ontario Medical Association [2]. As a result, people find themselves going to the ER with coughs and colds and flooding the ER causing massive wait times which ends in patients even resulting in leaving without being seen [3]. A massive part of the wait time is due to the overhead of documentation tasks. Doctors, healthcare professionals, and support staff find themselves spending most of their time on documentation which overall slows the pipeline of patients tremendously.

\subsection{Inputs and Outputs}
% \wss{Characterize the problem in terms of ``high level'' inputs and outputs.  
% Use abstraction so that you can avoid details.}
Inputs for this project will be the following:

\begin{itemize}
    \item \textbf{Doctor-Patient Conversation} -- The conversation between the doctor and the patient is recorded using the recording tool.
    \item \textbf{Medical Prediction Context} -- A comprehensive knowledge of common diseases and the associated medicines is served as input for the model to predict the diagnosis and treatment plans.
    \item \textbf{Text Field Input} -- Text field entries are utilized as input parameters provided by the user for various data processing operations such as filling out patient data, creating consultation notes, etc.
     
\end {itemize}

Outputs for this project will be the following: 

\begin{itemize} 
    \item \textbf{Transcribed Text} -- The audio from healthcare professional and patient conversation is transcribed into text.
    \item \textbf{Auto-filled documents} -- The transcribed text is further classified into relevant categories to autofill the charts with patient's medical information.
    \item \textbf{Predicted Diagnosis and Plan} -- Classified data is used to provide diagnosis and the treatment plan based on the context about common diseases and associated diagnosis.  
\end{itemize}

\wss{Inputs:}

\subsection{Stakeholders}

This project will include many stakeholders from developers, visionaries, supervisors, as well as adopters. Below are the stakeholders concerned with this project.

\begin{itemize}
\item \textbf{Application Users} -- These users will include the healthcare staff in the hospital using the application to speed up documentation throughout the patient journey. This will include receptionists, nurses, doctors as well as any other hospital members responsible for patient documentation. They will be the primary users as they are the target group whose time is most used up in patient documentation tasks.
\item \textbf{Development Team} -- This is the team of developers who will be implementing the solution. They will take whatever’s in the backlog and implement it as per the requirements into a functional and user-friendly application. This team will also include product owners as well as product managers who will turn the vision into a prioritized backlog. This will ensure the customer's needs are fulfilled in a systematic and agile fashion.
\item \textbf{Project Supervisors} -- These will include domain experts as well as potential users such as physicians. Project supervisors will be critical in envisioning the app. Namely assisting with requirements elicitation as well as helping in formulating functional and non-functional requirements. As members of the field, they can provide, review and give feedback on critical pain points and give feedback on the features to relieve the pain points.
\item \textbf{Application Clients} -- The bodies themselves that will purchase the software, in other words, this is essentially the health systems. These health systems will essentially plug this system into their network of hospitals where healthcare professionals can use this software.
\item \textbf{Regulatory Bodies} -- These are bodies that regulate patient health data. Since the whole patient journey will be tracked through the application abiding by security and privacy policy is a must. Examples of these bodies would be governments through policy, information and privacy commissions.
\end{itemize}

\subsection{Environment}
% \wss{Hardware and software environment}

The application will be implemented as a web application to ensure portability as wifi is a standard in clinics and hospitals. As per the environment, the app services will run in, that would be a cloud provider (AWS, GCP, Azure). 

\section{Goals}

The goals for this project are as follows:

\begin{table}[H]
    \centering
    \begin{tabular}{p{4cm} p{4cm} p{4cm}}
        \toprule
        \textbf{Goal} & \textbf{Explaining} & \textbf{Reasoning} \\
        \midrule
        Use voice to fill in medical documentation (charts, files, etc.) & The app will record conversations and automatically turn them into medical notes and charts. & This will save doctors time by automating paperwork, letting them focus more on patients. \\
        \midrule
        Reduce documentation overhead time.  & Through tracking the whole patient journey in the app, we look to reduce the overhead of triaging, clinical documentation, and other registrations.  & This helps hospital healthcare professionals focus on care and lowers the time taken through registration for hospital staff. \\
        \midrule
        Automated diagnosis suggestions.  & Use AI to suggest possible diagnoses based on what the doctor and patient discuss.  & This will help doctors make faster, more accurate diagnoses, especially in tricky cases. \\
        \midrule
        Automated treatment plan suggestions. & Based on diagnosis and patient data provide treatment plan suggestions. & This will help doctors fill out their charts faster. \\
        \bottomrule
    \end{tabular}
\end{table}

\section{Stretch Goals}

The stretch goals for this project are as follows:

\begin{table}[H]
    \centering
    \begin{tabular}{p{4cm} p{4cm} p{4cm}}
        \toprule
        \textbf{Goal} & \textbf{Explaining} & \textbf{Reasoning} \\ 
        \midrule
        Patient triage system. & Prioritize patients based on the severity of their condition. & Ensures the most critical patients receive care first, improving emergency response. \\
        \bottomrule
    \end{tabular}
\end{table}

\section{Challenge Level and Extras}

% \wss{State your expected challenge level (advanced, general or basic).  The
% challenge can come through the required domain knowledge, the implementation or
% something else.  Usually the greater the novelty of a project the greater its
% challenge level.  You should include your rationale for the selected level.
% Approval of the level will be part of the discussion with the instructor for
% approving the project.  The challenge level, with the approval (or request) of
% the instructor, can be modified over the course of the term.}

% \wss{Teams may wish to include extras as either potential bonus grades, or to
% make up for a less advanced challenge level.  Potential extras include usability
% testing, code walkthroughs, user documentation, formal proof, GenderMag
% personas, Design Thinking, etc.  Normally the maximum number of extras will be
% two.  Approval of the extras will be part of the discussion with the instructor
% for approving the project.  The extras, with the approval (or request) of the
% instructor, can be modified over the course of the term.}
In terms of the project challenge level, the project will come in the general category. This is supported by the reasoning below:

\begin{itemize}
    \item \textbf{Domain Knowledge} -- The documentation process has a lot of ins and outs which may differ between health organizations. Our supervisors and stakeholders will provide us with a base of the requirements, but further elicitation will be required to ensure that the requirements reflect a problem that truly exists. Additionally, since the whole patient journey is tracked we will have to survey other hospital staff as well to gain a further understanding.
    \item \textbf{Implementation Challenges} -- There will be quite a few microservices required for this project where each microservice has high complexity. The performance of the microservices must be high as this use case requires quick response time. Additionally, since we are dealing with patient data security and privacy must be upheld. Lastly, the integration between all of the parts must be secure and undergo rigorous integration testing.
\end{itemize}

As part of the extras for this project, we will accomplish the following extras:
\begin{itemize}
    \item Usability testing.
    \item User manual.
\end{itemize}

\section{References}
\begin{itemize}
  \item 
  \href{https://www.cbc.ca/news/canada/toronto/ontario-family-doctor-shortage-record-high-1.7261558 }{[1] Article on Family Doctor Shortage.}
  \item 
  \href{https://www.cbc.ca/news/canada/toronto/family-doctor-shortage-oma-1.7097935}{[2] Article on Doctor Shortage.}
  \item 
  \href{https://www.ices.on.ca/publications/journal-articles/association-between-waiting-times-and-short-term-mortality-and-hospital-admission-after-departure-from-the-emergency-department-population-based-cohort-study-from-ontario-canada/}{[3] Article on ER patients leaving without being seen.}
  \item
  \href{https://medicalguidelines.msf.org/sites/default/files/pdf/guideline-339-en.pdf}{[4] Report on essential medicines for diseases.}
  \item
  \href{https://www.paho.org/sites/default/files/Updated_sixteenth_adult_list_en.pdf}{[5] Report on essential medicines from World Health Organization.}
  \item
  \href{https://list.essentialmeds.org/files/trs/Png0UfvXyfT663AJYvAJdOVN9yGylap9OzVRqXY9.pdf}{[6] Report on the use of essential medicines from World Health Organization.}
\end {itemize}

\newpage{}

\section*{Appendix --- Reflection}

% \wss{Not required for CAS 741}

% \input{../Reflection.tex}

\begin{enumerate}
    \item \textbf{What went well while writing this deliverable?}\\\\
    This deliverable has allowed us to flesh out the rough ideas we had brainstormed. Through outlining the problem, stakeholders, inputs and outputs, we were able to provide a detailed insight into the problem. The deliverable also tested our knowledge and forced us to answer important parts of the problem.
    Next, writing out the goals and stretch goals really puts the direction on the course. This way the goals of the projects are clear.
    An additional benefit was the whole team could get assimilated with the issue tracker and the agile development. This has now set a precedent for how features should be contributed.

    \item \textbf{What pain points did you experience during this deliverable, and how did you resolve them?}\\\\
    Every team project has challenges which need to be resolved to move smoothly. To move smoothly, we needed to create a contribution strategy. We need to create a schedule where we have enough time to iteratively contribute and review each other’s work. Optimizing the schedule to create the best work possible was the hardest part. 
    Another issue was onboarding everyone as people in Computer Engineering have different schedules than Software program. However, distributing the workload and organizing meetings according to everyone’s availability has been a great source of help for the progress of the project.  

    \item \textbf{How did you and your team adjust the scope of your goals to ensure they are suitable for a Capstone project (not overly ambitious but also of appropriate complexity for a senior design project)?}\\\\
    Any adjustments in the goals were made by thinking if the solution we want to develop is a realistic one and would bring a positive change in society. After doing a thorough analysis of the goals, we decided to break them into smaller parts so that we don’t encounter any problems on a higher level. By not becoming over ambitious about the project and understanding the complexity of the design, we also sought feedback from our supervisor and the instructor regularly.
\end{enumerate}  

\end{document}