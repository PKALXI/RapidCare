\documentclass{article}

\usepackage{tabularx}
\usepackage{booktabs}

\title{Reflection and Traceability Report on \progname}

\author{\authname}

\date{}

\input{../Comments}
\input{../Common}

\begin{document}

\maketitle

\plt{Reflection is an important component of getting the full benefits from a
learning experience.  Besides the intrinsic benefits of reflection, this
document will be used to help the TAs grade how well your team responded to
feedback.  Therefore, traceability between Revision 0 and Revision 1 is and
important part of the reflection exercise.  In addition, several CEAB (Canadian
Engineering Accreditation Board) Learning Outcomes (LOs) will be assessed based
on your reflections.}

\section{Changes in Response to Feedback}

\plt{Summarize the changes made over the course of the project in response to
feedback from TAs, the instructor, teammates, other teams, the project
supervisor (if present), and from user testers.}

\plt{For those teams with an external supervisor, please highlight how the feedback 
from the supervisor shaped your project.  In particular, you should highlight the 
supervisor's response to your Rev 0 demonstration to them.}

\plt{Version control can make the summary relatively easy, if you used issues
and meaningful commits.  If you feedback is in an issue, and you responded in
the issue tracker, you can point to the issue as part of explaining your
changes.  If addressing the issue required changes to code or documentation, you
can point to the specific commit that made the changes.  Although the links are
helpful for the details, you should include a label for each item of feedback so
that the reader has an idea of what each item is about without the need to click
on everything to find out.}

\plt{If you were not organized with your commits, traceability between feedback
and commits will not be feasible to capture after the fact.  You will instead
need to spend time writing down a summary of the changes made in response to
each item of feedback.}

\plt{You should address EVERY item of feedback.  A table or itemized list is
recommended.  You should record every item of feedback, along with the source of
that feedback and the change you made in response to that feedback.  The
response can be a change to your documentation, code, or development process.
The response can also be the reason why no changes were made in response to the
feedback.  To make this information manageable, you will record the feedback and
response separately for each deliverable in the sections that follow.}

\plt{If the feedback is general or incomplete, the TA (or instructor) will not
be able to grade your response to feedback.  In that case your grade on this
document, and likely the Revision 1 versions of the other documents will be
low.} 

\subsection{SRS and Hazard Analysis}

\subsection{Design and Design Documentation}

\subsection{VnV Plan and Report}

\section{Challenge Level and Extras}

\subsection{Challenge Level}

The challenge level for the project is \textbf{General} as agreed upon by the course instructor. This classification perfectly reflects the project's scope and complexity.

\subsection{Extras}

The extras that were took by this project are usability testing and user manual.
In usability testing, the participants were given some task instructions to test the system. Post task, they were required to fill a survey to rate their experience with the system. The participants also gave some suggestions for future improvements in the system.
Additionally, a user manual is a technical document that is provided to assist people in using this project. It contains detailed description of each feature of the system as well as instructions on how to use it. 

\section{Design Iteration (LO11 (PrototypeIterate))}

The journey from the first version to the final version was a driven by the iterative feedback from the supervisor, TA, peers, other stakeholders, and the professor. Initially, goals were defined in the problem statement and the initial set of requirements was laid out in SRS. After meeting with the supervisor and hospital tour, we updated our goals and some requirements to introduce new features that will distinguish this project with the EHR that is being used currently in the healthcare industry. The Proof of Concept demo gave us a direction towards our final goal and we got an insight about how potential risks associated with the modules should be mitigated. The initial UI design was prepared in figma for both MG and MIS and the tests were prepared in VnVPlan. One of the changes that we did after the first version was the creation of AI-assist for the healthcare professional to query patient information. This addition is done after Rev 0 demo with our supervisor and is also a key distinction from the EHR system currently in use. Moreover, the diagnosis and treatment plan prediction from the audio conversation is another addition to the project to speed up the documentation process and save time. We also included Referrals and Prescriptions tabs in the system after meeting our supervisor before Rev 0 demo to further enhance the functionality of the system. Tests were updated and thorough testing was conducted to ensure new functionality meets the requirements. The usability surveys were updated to get feedback on new requirements and UI was enhanced as a result of usability testing.            

\section{Design Decisions (LO12)}

\plt{Reflect and justify your design decisions.  How did limitations,
 assumptions, and constraints influence your decisions?  Discuss each of these
 separately.}

\section{Economic Considerations (LO23)}

\plt{Is there a market for your product? What would be involved in marketing your 
product? What is your estimate of the cost to produce a version that you could 
sell?  What would you charge for your product?  How many units would you have to 
sell to make money? If your product isn't something that would be sold, like an 
open source project, how would you go about attracting users?  How many potential 
users currently exist?}

\section{Reflection on Project Management (LO24)}

\plt{This question focuses on processes and tools used for project management.}

\subsection{How Does Your Project Management Compare to Your Development Plan}

\plt{Did you follow your Development plan, with respect to the team meeting plan, 
team communication plan, team member roles and workflow plan.  Did you use the 
technology you planned on using?}

\subsection{What Went Well?}

\plt{What went well for your project management in terms of processes and 
technology?}

\subsection{What Went Wrong?}

\plt{What went wrong in terms of processes and technology?}

\subsection{What Would you Do Differently Next Time?}

\plt{What will you do differently for your next project?}

\section{Reflection on Capstone}

\plt{This question focuses on what you learned during the course of the capstone project.}

\subsection{Which Courses Were Relevant}

\plt{Which of the courses you have taken were relevant for the capstone project?}

\subsection{Knowledge/Skills Outside of Courses}

\plt{What skills/knowledge did you need to acquire for your capstone project
that was outside of the courses you took?}

\end{document}