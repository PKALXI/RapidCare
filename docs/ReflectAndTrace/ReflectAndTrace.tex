\documentclass{article}

\usepackage{tabularx}
\usepackage{booktabs}

\title{Reflection and Traceability Report on \progname}

\author{\authname}

\date{}

\input{../Comments}
\input{../Common}

\begin{document}

\maketitle

\plt{Reflection is an important component of getting the full benefits from a
learning experience.  Besides the intrinsic benefits of reflection, this
document will be used to help the TAs grade how well your team responded to
feedback.  Therefore, traceability between Revision 0 and Revision 1 is and
important part of the reflection exercise.  In addition, several CEAB (Canadian
Engineering Accreditation Board) Learning Outcomes (LOs) will be assessed based
on your reflections.}

\section{Changes in Response to Feedback}

\plt{Summarize the changes made over the course of the project in response to
feedback from TAs, the instructor, teammates, other teams, the project
supervisor (if present), and from user testers.}

\plt{For those teams with an external supervisor, please highlight how the feedback 
from the supervisor shaped your project.  In particular, you should highlight the 
supervisor's response to your Rev 0 demonstration to them.}

\plt{Version control can make the summary relatively easy, if you used issues
and meaningful commits.  If you feedback is in an issue, and you responded in
the issue tracker, you can point to the issue as part of explaining your
changes.  If addressing the issue required changes to code or documentation, you
can point to the specific commit that made the changes.  Although the links are
helpful for the details, you should include a label for each item of feedback so
that the reader has an idea of what each item is about without the need to click
on everything to find out.}

\plt{If you were not organized with your commits, traceability between feedback
and commits will not be feasible to capture after the fact.  You will instead
need to spend time writing down a summary of the changes made in response to
each item of feedback.}

\plt{You should address EVERY item of feedback.  A table or itemized list is
recommended.  You should record every item of feedback, along with the source of
that feedback and the change you made in response to that feedback.  The
response can be a change to your documentation, code, or development process.
The response can also be the reason why no changes were made in response to the
feedback.  To make this information manageable, you will record the feedback and
response separately for each deliverable in the sections that follow.}

\plt{If the feedback is general or incomplete, the TA (or instructor) will not
be able to grade your response to feedback.  In that case your grade on this
document, and likely the Revision 1 versions of the other documents will be
low.} 

\subsection{SRS and Hazard Analysis}

\subsection{Design and Design Documentation}

\subsection{VnV Plan and Report}

\section{Challenge Level and Extras}

\subsection{Challenge Level}

\plt{State the challenge level (advanced, general, basic) for your project.  Your challenge level should exactly match what is included in your problem statement.  This should be the challenge level agreed on between you and the course instructor.}

\subsection{Extras}

\plt{Summarize the extras (if any) that were tackled by this project.  Extras
can include usability testing, code walkthroughs, user documentation, formal
proof, GenderMag personas, Design Thinking, etc.  Extras should have already
been approved by the course instructor as included in your problem statement.}

\section{Design Iteration (LO11 (PrototypeIterate))}

\plt{Explain how you arrived at your final design and implementation.  How did
the design evolve from the first version to the final version?} 

\plt{Don't just say what you changed, say why you changed it.  The needs of the
client should be part of the explanation.  For example, if you made changes in
response to usability testing, explain what the testing found and what changes
it led to.}

\section{Design Decisions (LO12)}

\plt{Reflect and justify your design decisions.  How did limitations,
 assumptions, and constraints influence your decisions?  Discuss each of these
 separately.}

 \section{Economic Considerations (LO23)}

 There is a clear market for our product, RapidCare. The healthcare industry, particularly in Ontario, is facing an overwhelming documentation burden due to a shortage of family doctors, affecting over 2.5 million patients. Our solution automates healthcare documentation through voice-to-text transcription and ML-based diagnosis and medication suggestions, targeting hospitals and clinics to improve efficiency and reduce ER wait times. \\
 
 \noindent
 Marketing the product would involve outreach to healthcare networks and hospital systems, emphasizing the benefits of reduced documentation overhead, improved patient throughput, and clinician satisfaction. Usability testing and comprehensive user documentation are already part of the project deliverables to aid adoption. \\
 
 \noindent
 We have estimated the cost to produce and maintain a market-ready version at $\$500$ per month. Each subscription will place us in a net positive, making the business model financially sustainable from the first sale. \\
 
 \noindent
 Our current strategy involves direct outreach to hospital IT departments, leveraging existing connections to secure early adopters and drive initial growth. The potential user base includes doctors, nurses, and administrative staff across healthcare institutions, particularly those involved in clinical documentation. Given Ontario’s reported shortage and the size of its healthcare system, there are thousands of potential users.
 
 \section{Reflection on Project Management (LO24)}
 
 \subsection{How Does Your Project Management Compare to Your Development Plan}
 
 We followed the development plan closely. Team meetings were held weekly, with agendas, minutes, and action items tracked via GitHub issues. Microsoft Teams was used for communication, and GitHub Projects was used for documentation, code reviews, and task tracking. The technologies planned were all implemented as expected.
 
 \subsection{What Went Well?}
 
 \begin{itemize}
     \item Clear team roles and responsibilities helped maintain accountability.
     \item GitHub Projects provided an effective workflow.
     \item Code review and peer feedback ensured quality control.
 \end{itemize}
 
 \subsection{What Went Wrong?}
 
 \begin{itemize}
     \item Usability testing was not fully conducted due to time constraints.
     \item The variability in healthcare documentation between institutions posed elicitation challenges.
     \item Integration testing was limited during early phases and could have been started earlier.
 \end{itemize}
 
 \subsection{What Would You Do Differently Next Time?}
 
 \begin{itemize}
     \item Allocate more time for usability testing and stakeholder feedback.
     \item Increase stakeholder engagement throughout mid-development milestones.
 \end{itemize}
 
 \section{Reflection on Capstone}
 
 \subsection{Which Courses Were Relevant}
 
 The following courses were highly relevant to our capstone project:
 \begin{itemize}
     \item SWRENG 3DB3 - Databases.
     \item SFWRENG 4HC3 - Human-Computer Interfaces.
     \item SFWRENG 4AL3 - Applications of Machine Learning.
     \item ENGINEER 3PX3 - Integrated Engineering Design Project 3.
     \item SFWRENG 3A04 - Software Design III.
 \end{itemize}
 
 \subsection{Knowledge/Skills Outside of Courses}
 
 We had to acquire several skills beyond what was covered in our coursework:
 \begin{itemize}
     \item Domain-specific knowledge about healthcare documentation and medical terminology.
 \end{itemize}

\end{document}