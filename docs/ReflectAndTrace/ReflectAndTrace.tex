\documentclass{article}

\usepackage{tabularx}
\usepackage{booktabs}
\usepackage{geometry}
\usepackage{longtable}
\usepackage{hyperref}

\title{Reflection and Traceability Report on \progname}

\author{\authname}

\date{}

\input{../Comments}
\input{../Common}

\begin{document}

\maketitle

% \plt{Reflection is an important component of getting the full benefits from a
% learning experience.  Besides the intrinsic benefits of reflection, this
% document will be used to help the TAs grade how well your team responded to
% feedback.  Therefore, traceability between Revision 0 and Revision 1 is and
% important part of the reflection exercise.  In addition, several CEAB (Canadian
% Engineering Accreditation Board) Learning Outcomes (LOs) will be assessed based
% on your reflections.}

\section{Changes in Response to Feedback}

% \plt{Summarize the changes made over the course of the project in response to
% feedback from TAs, the instructor, teammates, other teams, the project
% supervisor (if present), and from user testers.}

% \plt{For those teams with an external supervisor, please highlight how the feedback 
% from the supervisor shaped your project.  In particular, you should highlight the 
% supervisor's response to your Rev 0 demonstration to them.}

% \plt{Version control can make the summary relatively easy, if you used issues
% and meaningful commits.  If you feedback is in an issue, and you responded in
% the issue tracker, you can point to the issue as part of explaining your
% changes.  If addressing the issue required changes to code or documentation, you
% can point to the specific commit that made the changes.  Although the links are
% helpful for the details, you should include a label for each item of feedback so
% that the reader has an idea of what each item is about without the need to click
% on everything to find out.}

% \plt{If you were not organized with your commits, traceability between feedback
% and commits will not be feasible to capture after the fact.  You will instead
% need to spend time writing down a summary of the changes made in response to
% each item of feedback.}

% \plt{You should address EVERY item of feedback.  A table or itemized list is
% recommended.  You should record every item of feedback, along with the source of
% that feedback and the change you made in response to that feedback.  The
% response can be a change to your documentation, code, or development process.
% The response can also be the reason why no changes were made in response to the
% feedback.  To make this information manageable, you will record the feedback and
% response separately for each deliverable in the sections that follow.}

% \plt{If the feedback is general or incomplete, the TA (or instructor) will not
% be able to grade your response to feedback.  In that case your grade on this
% document, and likely the Revision 1 versions of the other documents will be
% low.} 

This section summarizes the changes we have made in regards to the feedback we got from TAs, supervisor, peer review, supervisor, and during usability testing. 
The changes made can be found in the below able by clicking on associated issues. 
In order to ease the traceability we have categorized the issue into milestones such as 'TA Feedback Issues', 'Peer Review', and 'Final Doc Updates' which can be found in \href{https://github.com/users/Inreet-Kaur/projects/2/views/1}{Team-25's Capstone Deliverables Project}. Associated PRs for issues can be viewed under specific views created for \href{https://github.com/users/Inreet-Kaur/projects/2/views/8?sliceBy%5Bvalue%5D=TA+Feedback+Issues}{TA feedback},  \href{https://github.com/users/Inreet-Kaur/projects/2/views/10?sliceBy%5Bvalue%5D=Peer+Review}{Peer review}, and \href{https://github.com/users/Inreet-Kaur/projects/2/views/9?sliceBy%5Bvalue%5D=Milestone+8+-+Final+Doc+Updates}{Final documentation updates} issues.

\subsection{SRS and Hazard Analysis}

Changes to SRS and Hazard Analysis along with the feedback, response and associated issues and PRs can be found in the tables below: 

\begin{longtable}{| p{0.1\textwidth} | p{0.4\textwidth} | p{0.3\textwidth} | p{0.1\textwidth} |}
    \caption{Changes for SRS Documentation} \\
    \hline
    \textbf{Feedback Source} & \textbf{Feedback Item} & \textbf{Response} & \textbf{Issue} \\
    \hline
    \endfirsthead
    \hline
    \textbf{Feedback Source} & \textbf{Feedback Item} & \textbf{Response} & \textbf{Issue} \\
    \hline
    \endhead
    \hline
    \endfoot
    TA & Document Content: Missing revision history; \newline Document Organization: Incorrect template; \newline LO Ext. Knowledge: No mention of feedback from supervisor; \newline LO Formalization: Missing formalization and other sections related to data types, etc. & Template updated to include all relevant sections, including formalization. External knowledge from supervisor added. Revision history updated. & \href{https://github.com/users/Inreet-Kaur/projects/2/views/8?sliceBy%5Bvalue%5D=TA+Feedback+Issues&pane=issue&itemId=92154285&issue=PKALXI%7CRapidCare%7C166}{\#166} \\
    \hline
    TA & Formatting and Style: Grammar and capitalization errors; \newline Focus on Users: Issues with stakeholder and user section; \newline Phase In Plan: Missing phase-in plan; \newline LO Impact: Impact on society is not clear. & Fixed grammar and capitalization errors. Updated stakeholders and users section. Society is listed as a stakeholder along with impacts on health, safety, cultural diversity, etc. Phase-in plan added. & \href{https://github.com/users/Inreet-Kaur/projects/2/views/8?sliceBy%5Bvalue%5D=TA+Feedback+Issues&pane=issue&itemId=92154311&issue=PKALXI%7CRapidCare%7C167}{\#167} \\
    \hline
    TA & Notations and Conventions: Use case diagram missing labels. & Use case diagram updated. & \href{https://github.com/users/Inreet-Kaur/projects/2/views/8?sliceBy%5Bvalue%5D=TA+Feedback+Issues&pane=issue&itemId=92154354&issue=PKALXI%7CRapidCare%7C168}{\#168} \\
    \hline
    TA & What not How (Abstract): Constraints listed as requirements; \newline Basis for Design: Vague constraints, HIPAA included which is not applicable; \newline LO Standards: Vague requirements. & Constraints and NFRs updated to address issues. Compliance is changed to PIPEDA. & \href{https://github.com/users/Inreet-Kaur/projects/2/views/8?sliceBy%5Bvalue%5D=TA+Feedback+Issues&pane=issue&itemId=92154366&issue=PKALXI%7CRapidCare%7C169}{\#169} \\
    \hline
    TA & Complete, Correct, and Unambiguous: Unclear requirements; \newline Traceable Requirements: Incorrect traceability matrix; \newline Verifiable Requirements: Unclear fit criterion. & Outlined requirements and reworded fit criterion. Traceability matrix fixed. & \href{https://github.com/users/Inreet-Kaur/projects/2/views/8?sliceBy%5Bvalue%5D=TA+Feedback+Issues&pane=issue&itemId=92154380&issue=PKALXI%7CRapidCare%7C170}{\#170} \\
    \hline
    Peer Review & Unclear Requirements for Accessibility Compliance. & Requirement removed as a result of changed scope. & \href{https://github.com/PKALXI/RapidCare/issues/79}{\#79} \\
    \hline
    Peer Review & Missing Phase-in Plan. & Added as a result of TA feedback. & \href{https://github.com/PKALXI/RapidCare/issues/80}{\#80} \\
    \hline
    Peer Review & Vague Specification for Functional Requirements. & Requirements reworded. & \href{https://github.com/PKALXI/RapidCare/issues/81}{\#81} \\
    \hline
    Peer Review & Complete, Correct, and Unambiguous Criteria: Functional Requirements not properly worded. & Reworded functional requirements. & \href{https://github.com/PKALXI/RapidCare/issues/82}{\#82} \\
    \hline
    Peer Review & NFRs: Gap in Data Backup and Recovery. & NFRs reworded and updated. & \href{https://github.com/PKALXI/RapidCare/issues/83}{\#83} \\
    \hline
    Peer Review & Requirements for System Scalability lack details. & NFR reworded and updated. & \href{https://github.com/PKALXI/RapidCare/issues/84}{\#84} \\
    \hline
\end{longtable}

\begin{longtable}{| p{0.1\textwidth} | p{0.4\textwidth} | p{0.3\textwidth} | p{0.1\textwidth} |}
    \caption{Changes for Hazard Analysis} \\
    \hline
    \textbf{Feedback Source} & \textbf{Feedback Item} & \textbf{Response} & \textbf{Issue} \\
    \hline
    \endfirsthead
    \hline
    \textbf{Feedback Source} & \textbf{Feedback Item} & \textbf{Response} & \textbf{Issue} \\
    \hline
    \endhead
    \hline
    \endfoot
    TA & Spelling, grammar, and style: FMEA table formatting. & Formatting fixed. & \href{https://github.com/users/Inreet-Kaur/projects/2/views/8?sliceBy%5Bvalue%5D=TA+Feedback+Issues&pane=issue&itemId=92154633&issue=PKALXI%7CRapidCare%7C171}{\#171}\\
    \hline
    TA & Recommended Actions: Mention user involvement in assumptions. & Assumptions updated. & \href{https://github.com/users/Inreet-Kaur/projects/2/views/8?sliceBy%5Bvalue%5D=TA+Feedback+Issues&pane=issue&itemId=92154648&issue=PKALXI%7CRapidCare%7C172}{\#172} \\
    \hline
    Peer Review & Missing Requirements: Missing some of the requirements such as security, etc. & Requirements included in SRS. & \href{https://github.com/PKALXI/RapidCare/issues/103}{\#103} \\
    \hline
    Peer Review & Scope and Purpose of Hazard Analysis: API modules need more robust solutions. & Details of robust solutions added. & \href{https://github.com/PKALXI/RapidCare/issues/104}{\#104} \\
    \hline
    Peer Review & FMEA table does not account for formatting issues in classification. & Potential hazard added in the table. & \href{https://github.com/PKALXI/RapidCare/issues/105}{\#105} \\
    \hline
    Peer Review & Roadmap lacks clarity. & Roadmap updated for clarity to include how iterative feedback would be used. & \href{https://github.com/PKALXI/RapidCare/issues/106}{\#106} \\
    \hline
    Peer Review & Access requirements do not address how the system will handle repeated failed login attempts. & Access requirements reworded to handle repeated login attempts. & \href{https://github.com/PKALXI/RapidCare/issues/107}{\#107} \\
    \hline
    Peer Review & Detection gaps in FMEA table. & Measures added for data accuracy and consistency. & \href{https://github.com/PKALXI/RapidCare/issues/108}{\#108} \\
    \hline
\end{longtable}

\subsection{Design and Design Documentation}

Changes to Desgin and Design Documentation along with the feedback, response and associated issues and PRs can be found in the tables below: 

\begin{longtable}{| p{0.1\textwidth} | p{0.4\textwidth} | p{0.3\textwidth} | p{0.1\textwidth} |}
    \caption{Changes for Design} \\
    \hline
    \textbf{Feedback Source} & \textbf{Feedback Item} & \textbf{Response} & \textbf{Issue} \\
    \hline
    \endfirsthead
    \hline
    \textbf{Feedback Source} & \textbf{Feedback Item} & \textbf{Response} & \textbf{Issue} \\
    \hline
    \endhead
    \hline
    \endfoot
    Peer Review&  &  & \href{}{} \\
    \hline
    Peer Review&  &  & \href{}{} \\
    \hline
    Peer Review&  &  & \href{}{} \\
    \hline
    Peer Review&  &  & \href{}{} \\
    \hline
    Peer Review&  &  & \href{}{} \\
    \hline
    Peer Review&  &  & \href{}{} \\
    \hline
\end{longtable}

\begin{longtable}{| p{0.1\textwidth} | p{0.4\textwidth} | p{0.3\textwidth} | p{0.1\textwidth} |}
    \caption{Changes for MG and MIS} \\
    \hline
    \textbf{Feedback Source} & \textbf{Feedback Item} & \textbf{Response} & \textbf{Issue} \\
    \hline
    \endfirsthead
    \hline
    \textbf{Feedback Source} & \textbf{Feedback Item} & \textbf{Response} & \textbf{Issue} \\
    \hline
    \endhead
    \hline
    \endfoot
    TA & SoftArchitec: Quality Information: Some secrets are not secrets and should be updated. & kalsi & \href{https://github.com/users/Inreet-Kaur/projects/2/views/1?pane=issue&itemId=104763444&issue=PKALXI%7CRapidCare%7C419}{\#419} \\
    \hline
    TA & DetDesDoc: EnoughToBuild: Vague and ambious wording. & kalsi & \href{https://github.com/users/Inreet-Kaur/projects/2/views/1?pane=issue&itemId=104763999&issue=PKALXI%7CRapidCare%7C420}{\#420} \\
    \hline
    TA & CI/CD Infrastruture: Not working, actions failing. & CI/CD fully setup with no issues. & \href{https://github.com/users/Inreet-Kaur/projects/2/views/1?pane=issue&itemId=104764205&issue=PKALXI%7CRapidCare%7C421}{\#421} \\
    \hline
    TA & LO SpecMath: No specific feedback provided. & Already formalised best to our ability. & \href{https://github.com/users/Inreet-Kaur/projects/2/views/1?pane=issue&itemId=104764560&issue=PKALXI%7CRapidCare%7C422}{\#422}\\
    \hline
    TA & LO ProbSolutions: No specific feedback provided. & Already completed best to our ability. & \href{https://github.com/users/Inreet-Kaur/projects/2/views/1?pane=issue&itemId=104764784&issue=PKALXI%7CRapidCare%7C423}{\#423} \\
    \hline
    TA & LO Explores: No specific feedback provided. & Already completed best to our ability. & \href{https://github.com/users/Inreet-Kaur/projects/2/views/1?pane=issue&itemId=104765036&issue=PKALXI%7CRapidCare%7C424}{\#424} \\
    \hline
    Peer Review & Module Guide: Secrets for patient model and administrator model module do not cover data validation, consistency rules, and any internal logic hidden from other modules. & Minor changes added, main secret is mentioned, data validation is a inferred function as well. & \href{https://github.com/PKALXI/RapidCare/issues/249}{\#249} \\
    \hline
    Peer Review & MG: Insufficient Detail on Relationships Between Modules.  & Provided in the network sections, HTTP + sockets. & \href{https://github.com/PKALXI/RapidCare/issues/250}{\#250} \\
    \hline
    Peer Review & MIS: Missing Details in Assumptions for Prediction Modules. & Updated assumptions to include edge cases, such as handling missing or invalid data in the input chart. & \href{https://github.com/PKALXI/RapidCare/issues/251}{\#251} \\
    \hline
    Peer Review & MIS: Missing Error Handling Details in Broker Module. & Added details on how the Broker Module handles module-level failures. & \href{https://github.com/PKALXI/RapidCare/issues/252}{\#252} \\
    \hline
    Peer Review & MG: Lack of Integration Details for Administrator Account Management. & Included a description of the request/response structure for the CRUD operations and specify validation requirements for API calls. & \href{https://github.com/PKALXI/RapidCare/issues/253}{\#253} \\
    \hline
\end{longtable}

\subsection{VnV Plan and Report}

Changes to VnV Plan and Report along with the feedback, response and associated issues and PRs can be found in the table below: 

\begin{longtable}{| p{0.1\textwidth} | p{0.4\textwidth} | p{0.3\textwidth} | p{0.1\textwidth} |}
    \caption{Changes for VnV Plan} \\
    \hline
    \textbf{Feedback Source} & \textbf{Feedback Item} & \textbf{Response} & \textbf{Issue} \\
    \hline
    \endfirsthead
    \hline
    \textbf{Feedback Source} & \textbf{Feedback Item} & \textbf{Response} & \textbf{Issue} \\
    \hline
    \endhead
    \hline
    \endfoot
    TA & Content: Broken entry in references & Reference issue fixed. & \href{https://github.com/users/Inreet-Kaur/projects/2/views/8?sliceBy%5Bvalue%5D=TA+Feedback+Issues&pane=issue&itemId=93054421&issue=PKALXI%7CRapidCare%7C190}{\#190}\\
    \hline
    TA & Spelling and grammar and style: Break into paragraphs such that one paragraph talks about one topic. & Formatting fixed. & \href{https://github.com/users/Inreet-Kaur/projects/2/views/8?sliceBy%5Bvalue%5D=TA+Feedback+Issues&pane=issue&itemId=93054661&issue=PKALXI%7CRapidCare%7C191}{\#191} \\
    \hline
    TA & Plan: VnV Reviews issues. & Updated to remove mutation testing and include a checklist. & \href{https://github.com/users/Inreet-Kaur/projects/2/views/8?sliceBy%5Bvalue%5D=TA+Feedback+Issues&pane=issue&itemId=93054750&issue=PKALXI%7CRapidCare%7C194}{\#194}\\
    \hline
    TA & System Tests for Functional Requirements are specific: Issues in input and error messages. & Provide concrete inputs and error messages. & \href{https://github.com/users/Inreet-Kaur/projects/2/views/8?sliceBy%5Bvalue%5D=TA+Feedback+Issues&pane=issue&itemId=93056576&issue=PKALXI%7CRapidCare%7C197}{\#197}\\
    \hline
    TA & Tests for Nonfunctional Requirements are specific: Vague tests. & Updated tests to include details and concrete details. & \href{https://github.com/users/Inreet-Kaur/projects/2/views/8?sliceBy%5Bvalue%5D=TA+Feedback+Issues&pane=issue&itemId=93056734&issue=PKALXI%7CRapidCare%7C198}{\#198}\\
    \hline
    TA & Nondynamic testing used as necessary: Details missing for static testing. & Added details for static testing and fixed errors. & \href{https://github.com/users/Inreet-Kaur/projects/2/views/8?sliceBy%5Bvalue%5D=TA+Feedback+Issues&pane=issue&itemId=93060932&issue=PKALXI%7CRapidCare%7C200}{\#200}\\
    \hline
    Peer Review & General Infromation: Lacks clarity in its objectives. & Updated to emphasize the critical nature of safety and security in healthcare applications.  & \href{https://github.com/PKALXI/RapidCare/issues/137}{\#137} \\
    \hline
    Peer Review & Usability survey: Does not include specific and open-ended questions. & Usability survey pdated to include tailored questions to collect data on various design components. & \href{https://github.com/PKALXI/RapidCare/issues/138}{\#138} \\
    \hline
    Peer Review & Implementation Verification Plan: Does not provide clear criteria for identifying "critical sections". & Updated to include checklist to identify critical sections. & \href{https://github.com/PKALXI/RapidCare/issues/139}{\#139} \\
    \hline
    Peer Review & Tests for functional requirements: Lacks details on error messages details. & Updated output to include specific details in error messages. & \href{https://github.com/PKALXI/RapidCare/issues/140}{\#140} \\
    \hline
    Peer Review & Static Testing Procedures: Lacks structured apporach. & Static Testing Procedures updated to include structured approach and a checklist. & \href{https://github.com/PKALXI/RapidCare/issues/141}{\#141} \\
    \hline
    Peer Review & Software Validation Plan: Lacks specific criteria for validating the software against stakeholder expectations.  & Updated to include specific criteria to validate software. & \href{https://github.com/PKALXI/RapidCare/issues/142}{\#142} \\
    \hline

\end{longtable}

\begin{longtable}{| p{0.1\textwidth} | p{0.4\textwidth} | p{0.3\textwidth} | p{0.1\textwidth} |}
    \caption{Changes for VnV Report} \\
    \hline
    \textbf{Feedback Source} & \textbf{Feedback Item} & \textbf{Response} & \textbf{Issue} \\
    \hline
    \endfirsthead
    \hline
    \textbf{Feedback Source} & \textbf{Feedback Item} & \textbf{Response} & \textbf{Issue} \\
    \hline
    \endhead
    \hline
    \endfoot
    Peer Review & VnV Report: Verification of Nonfunctional Requirement 1 is missing & Results from usability testing included. & \href{https://github.com/PKALXI/RapidCare/issues/357}{\#357} \\
    \hline
    Peer Review & Report has grammar and spelling issues. & Updated to fix grammar and spellings. & \href{https://github.com/PKALXI/RapidCare/issues/358}{\#358} \\
    \hline
    Peer Review & VnV Report Insufficient Explanation for Deviations from the VnV Plan.  & Both VnV PLan and Report updated to include changes due to updated requirements. Any deviations are fully explained. & \href{https://github.com/PKALXI/RapidCare/issues/359}{\#359} \\
    \hline
    Peer Review & VnV Report Shallow Safety and Security Testing. & Updated Safety and security requirements to include specific creteria and match updated requirements. & \href{https://github.com/PKALXI/RapidCare/issues/360}{\#360} \\
    \hline
    Peer Review & VnV Report: Revision history not updated. & Revision history updated to include details. & \href{https://github.com/PKALXI/RapidCare/issues/361}{\#361} \\
    \hline
    Peer Review & VnV Report: Undefined Pass/Fail Criteria for Voice-to-Text Transcription & Tests updated to include specifics. Also socvered as a part of unit testing. & \href{https://github.com/PKALXI/RapidCare/issues/362}{\#362} \\
    \hline
\end{longtable}

\section{Challenge Level and Extras}

\subsection{Challenge Level}

\plt{State the challenge level (advanced, general, basic) for your project.  Your challenge level should exactly match what is included in your problem statement.  This should be the challenge level agreed on between you and the course instructor.}

\subsection{Extras}

\plt{Summarize the extras (if any) that were tackled by this project.  Extras
can include usability testing, code walkthroughs, user documentation, formal
proof, GenderMag personas, Design Thinking, etc.  Extras should have already
been approved by the course instructor as included in your problem statement.}

\section{Design Iteration (LO11 (PrototypeIterate))}

\plt{Explain how you arrived at your final design and implementation.  How did
the design evolve from the first version to the final version?} 

\plt{Don't just say what you changed, say why you changed it.  The needs of the
client should be part of the explanation.  For example, if you made changes in
response to usability testing, explain what the testing found and what changes
it led to.}

\section{Design Decisions (LO12)}

\plt{Reflect and justify your design decisions.  How did limitations,
 assumptions, and constraints influence your decisions?  Discuss each of these
 separately.}

\section{Economic Considerations (LO23)}

\plt{Is there a market for your product? What would be involved in marketing your 
product? What is your estimate of the cost to produce a version that you could 
sell?  What would you charge for your product?  How many units would you have to 
sell to make money? If your product isn't something that would be sold, like an 
open source project, how would you go about attracting users?  How many potential 
users currently exist?}

\section{Reflection on Project Management (LO24)}

\plt{This question focuses on processes and tools used for project management.}

\subsection{How Does Your Project Management Compare to Your Development Plan}

\plt{Did you follow your Development plan, with respect to the team meeting plan, 
team communication plan, team member roles and workflow plan.  Did you use the 
technology you planned on using?}

\subsection{What Went Well?}

\plt{What went well for your project management in terms of processes and 
technology?}

\subsection{What Went Wrong?}

\plt{What went wrong in terms of processes and technology?}

\subsection{What Would you Do Differently Next Time?}

\plt{What will you do differently for your next project?}

\section{Reflection on Capstone}

\plt{This question focuses on what you learned during the course of the capstone project.}

\subsection{Which Courses Were Relevant}

\plt{Which of the courses you have taken were relevant for the capstone project?}

\subsection{Knowledge/Skills Outside of Courses}

\plt{What skills/knowledge did you need to acquire for your capstone project
that was outside of the courses you took?}

\end{document}