\documentclass{article}

\usepackage{tabularx}
\usepackage{booktabs}
\usepackage{geometry}
\usepackage{longtable}
\usepackage{hyperref}

\title{Reflection and Traceability Report on \progname}

\author{\authname}

\date{}

\input{../Comments}
\input{../Common}

\begin{document}

\maketitle

% \plt{Reflection is an important component of getting the full benefits from a
% learning experience.  Besides the intrinsic benefits of reflection, this
% document will be used to help the TAs grade how well your team responded to
% feedback.  Therefore, traceability between Revision 0 and Revision 1 is and
% important part of the reflection exercise.  In addition, several CEAB (Canadian
% Engineering Accreditation Board) Learning Outcomes (LOs) will be assessed based
% on your reflections.}

\section{Changes in Response to Feedback}

% \plt{Summarize the changes made over the course of the project in response to
% feedback from TAs, the instructor, teammates, other teams, the project
% supervisor (if present), and from user testers.}

% \plt{For those teams with an external supervisor, please highlight how the feedback 
% from the supervisor shaped your project.  In particular, you should highlight the 
% supervisor's response to your Rev 0 demonstration to them.}

% \plt{Version control can make the summary relatively easy, if you used issues
% and meaningful commits.  If you feedback is in an issue, and you responded in
% the issue tracker, you can point to the issue as part of explaining your
% changes.  If addressing the issue required changes to code or documentation, you
% can point to the specific commit that made the changes.  Although the links are
% helpful for the details, you should include a label for each item of feedback so
% that the reader has an idea of what each item is about without the need to click
% on everything to find out.}

% \plt{If you were not organized with your commits, traceability between feedback
% and commits will not be feasible to capture after the fact.  You will instead
% need to spend time writing down a summary of the changes made in response to
% each item of feedback.}

% \plt{You should address EVERY item of feedback.  A table or itemized list is
% recommended.  You should record every item of feedback, along with the source of
% that feedback and the change you made in response to that feedback.  The
% response can be a change to your documentation, code, or development process.
% The response can also be the reason why no changes were made in response to the
% feedback.  To make this information manageable, you will record the feedback and
% response separately for each deliverable in the sections that follow.}

% \plt{If the feedback is general or incomplete, the TA (or instructor) will not
% be able to grade your response to feedback.  In that case your grade on this
% document, and likely the Revision 1 versions of the other documents will be
% low.} 

This section summarizes the changes we have made in regards to the feedback we got from TAs, supervisor, peer review, supervisor, and during usability testing. 
The changes made can be found in the below able by clicking on associated issues. 
In order to ease the traceability we have categorized the issue into milestones such as 'TA Feedback Issues', 'Peer Review', and 'Final Doc Updates' which can be found in \href{https://github.com/users/Inreet-Kaur/projects/2/views/1}{Team-25's Capstone Deliverables Project}. Associated PRs for issues can be viewed under specific views created for \href{https://github.com/users/Inreet-Kaur/projects/2/views/8?sliceBy%5Bvalue%5D=TA+Feedback+Issues}{TA feedback},  \href{https://github.com/users/Inreet-Kaur/projects/2/views/10?sliceBy%5Bvalue%5D=Peer+Review}{Peer review}, and \href{https://github.com/users/Inreet-Kaur/projects/2/views/9?sliceBy%5Bvalue%5D=Milestone+8+-+Final+Doc+Updates}{Final documentation updates} issues.

\subsection{SRS and Hazard Analysis}

Changes to SRS and Hazard Analysis along with the feedback, response, and associated issues can be found in the tables below: 

\begin{longtable}{| p{0.1\textwidth} | p{0.4\textwidth} | p{0.3\textwidth} | p{0.1\textwidth} |}
    \caption{Changes for SRS Documentation} \\
    \hline
    \textbf{Feedback Source} & \textbf{Feedback Item} & \textbf{Response} & \textbf{Issue} \\
    \hline
    \endfirsthead
    \hline
    \textbf{Feedback Source} & \textbf{Feedback Item} & \textbf{Response} & \textbf{Issue} \\
    \hline
    \endhead
    \hline
    \endfoot
    TA & Document Content: Missing revision history; \newline Document Organization: Incorrect template; \newline LO Ext. Knowledge: No mention of feedback from supervisor; \newline LO Formalization: Missing formalization and other sections related to data types, etc. & Template updated to include all relevant sections, including formalization. External knowledge from supervisor added. Revision history updated. & \href{https://github.com/users/Inreet-Kaur/projects/2/views/8?sliceBy%5Bvalue%5D=TA+Feedback+Issues&pane=issue&itemId=92154285&issue=PKALXI%7CRapidCare%7C166}{\#166} \\
    \hline
    TA & Formatting and Style: Grammar and capitalization errors; \newline Focus on Users: Issues with stakeholder and user section; \newline Phase In Plan: Missing phase-in plan; \newline LO Impact: Impact on society is not clear. & Fixed grammar and capitalization errors. Updated stakeholders and users section. Society is listed as a stakeholder along with impacts on health, safety, cultural diversity, etc. Phase-in plan added. & \href{https://github.com/users/Inreet-Kaur/projects/2/views/8?sliceBy%5Bvalue%5D=TA+Feedback+Issues&pane=issue&itemId=92154311&issue=PKALXI%7CRapidCare%7C167}{\#167} \\
    \hline
    TA & Notations and Conventions: Use case diagram missing labels. & Use case diagram updated. & \href{https://github.com/users/Inreet-Kaur/projects/2/views/8?sliceBy%5Bvalue%5D=TA+Feedback+Issues&pane=issue&itemId=92154354&issue=PKALXI%7CRapidCare%7C168}{\#168} \\
    \hline
    TA & What not How (Abstract): Constraints listed as requirements; \newline Basis for Design: Vague constraints, HIPAA included which is not applicable; \newline LO Standards: Vague requirements. & Constraints and NFRs updated to address issues. Compliance is changed to PIPEDA. & \href{https://github.com/users/Inreet-Kaur/projects/2/views/8?sliceBy%5Bvalue%5D=TA+Feedback+Issues&pane=issue&itemId=92154366&issue=PKALXI%7CRapidCare%7C169}{\#169} \\
    \hline
    TA & Complete, Correct, and Unambiguous: Unclear requirements; \newline Traceable Requirements: Incorrect traceability matrix; \newline Verifiable Requirements: Unclear fit criterion. & Outlined requirements and reworded fit criterion. Traceability matrix fixed. & \href{https://github.com/users/Inreet-Kaur/projects/2/views/8?sliceBy%5Bvalue%5D=TA+Feedback+Issues&pane=issue&itemId=92154380&issue=PKALXI%7CRapidCare%7C170}{\#170} \\
    \hline
    Peer Review & Unclear Requirements for Accessibility Compliance. & Requirement removed as a result of changed scope. & \href{https://github.com/PKALXI/RapidCare/issues/79}{\#79} \\
    \hline
    Peer Review & Missing Phase-in Plan. & Added as a result of TA feedback. & \href{https://github.com/PKALXI/RapidCare/issues/80}{\#80} \\
    \hline
    Peer Review & Vague Specification for Functional Requirements. & Requirements reworded. & \href{https://github.com/PKALXI/RapidCare/issues/81}{\#81} \\
    \hline
    Peer Review & Complete, Correct, and Unambiguous Criteria: Functional Requirements not properly worded. & Reworded functional requirements. & \href{https://github.com/PKALXI/RapidCare/issues/82}{\#82} \\
    \hline
    Peer Review & NFRs: Gap in Data Backup and Recovery. & NFRs reworded and updated. & \href{https://github.com/PKALXI/RapidCare/issues/83}{\#83} \\
    \hline
    Peer Review & Requirements for System Scalability lack details. & NFR reworded and updated. & \href{https://github.com/PKALXI/RapidCare/issues/84}{\#84} \\
    \hline
    Design Changes & NA & Requirements for AI Assist, feedback messages, and other design changes were added. & \href{https://github.com/users/Inreet-Kaur/projects/2/views/9?pane=issue&itemId=104550695&issue=PKALXI%7CRapidCare%7C402}{\#402} \\
    \hline
\end{longtable}

\begin{longtable}{| p{0.1\textwidth} | p{0.4\textwidth} | p{0.3\textwidth} | p{0.1\textwidth} |}
    \caption{Changes for Hazard Analysis} \\
    \hline
    \textbf{Feedback Source} & \textbf{Feedback Item} & \textbf{Response} & \textbf{Issue} \\
    \hline
    \endfirsthead
    \hline
    \textbf{Feedback Source} & \textbf{Feedback Item} & \textbf{Response} & \textbf{Issue} \\
    \hline
    \endhead
    \hline
    \endfoot
    TA & Spelling, grammar, and style: FMEA table formatting. & Formatting fixed. & \href{https://github.com/users/Inreet-Kaur/projects/2/views/8?sliceBy%5Bvalue%5D=TA+Feedback+Issues&pane=issue&itemId=92154633&issue=PKALXI%7CRapidCare%7C171}{\#171}\\
    \hline
    TA & Recommended Actions: Mention user involvement in assumptions. & Assumptions updated. & \href{https://github.com/users/Inreet-Kaur/projects/2/views/8?sliceBy%5Bvalue%5D=TA+Feedback+Issues&pane=issue&itemId=92154648&issue=PKALXI%7CRapidCare%7C172}{\#172} \\
    \hline
    Peer Review & Missing Requirements: Missing some of the requirements such as security, etc. & Requirements included in SRS. & \href{https://github.com/PKALXI/RapidCare/issues/103}{\#103} \\
    \hline
    Peer Review & Scope and Purpose of Hazard Analysis: API modules need more robust solutions. & Details of robust solutions added. & \href{https://github.com/PKALXI/RapidCare/issues/104}{\#104} \\
    \hline
    Peer Review & FMEA table does not account for formatting issues in classification. & Potential hazard added in the table. & \href{https://github.com/PKALXI/RapidCare/issues/105}{\#105} \\
    \hline
    Peer Review & Roadmap lacks clarity. & Roadmap updated for clarity to include how iterative feedback would be used. & \href{https://github.com/PKALXI/RapidCare/issues/106}{\#106} \\
    \hline
    Peer Review & Access requirements do not address how the system will handle repeated failed login attempts. & Access requirements reworded to handle repeated login attempts. & \href{https://github.com/PKALXI/RapidCare/issues/107}{\#107} \\
    \hline
    Peer Review & Detection gaps in FMEA table. & Measures added for data accuracy and consistency. & \href{https://github.com/PKALXI/RapidCare/issues/108}{\#108} \\
    \hline
    Design Changes & NA & Components updated and usability survey updated according to design changes. & \href{https://github.com/users/Inreet-Kaur/projects/2/views/9?pane=issue&itemId=104550956&issue=PKALXI%7CRapidCare%7C403}{\#403} \\
    \hline
\end{longtable}

\subsection{Design and Design Documentation}

Changes to Design and Design Documentation along with the feedback, response, and associated issues can be found in the tables below: 

\begin{longtable}{| p{0.1\textwidth} | p{0.4\textwidth} | p{0.3\textwidth} | p{0.1\textwidth} |}
    \caption{Changes for Design} \\
    \hline
    \textbf{Feedback Source} & \textbf{Feedback Item} & \textbf{Response} & \textbf{Issue} \\
    \hline
    \endfirsthead
    \hline
    \textbf{Feedback Source} & \textbf{Feedback Item} & \textbf{Response} & \textbf{Issue} \\
    \hline
    \endhead
    \hline
    \endfoot
    Supervisor & Add functionality for uploading and creating prescriptions. & Created the functionality to create prescriptions. The feature was not fully developed as per feedback in Rev 0 demonstration to prioritize more important aspects of the project. & \href{https://github.com/users/Inreet-Kaur/projects/6/views/2?pane=issue&itemId=97521959&issue=PKALXI%7CRapidCare%7C295}{\#295} \\
    \hline
    Supervisor & Add functionality for uploading and creating referrals. & Created the functionality to create referrals. The feature was not fully developed as per feedback in Rev 0 demonstration to prioritize more important aspects of the project. & \href{https://github.com/users/Inreet-Kaur/projects/6/views/2?pane=issue&itemId=97521127&issue=PKALXI%7CRapidCare%7C294}{\#294} \\
    \hline
    Supervisor, TA, Professor & Functionality to query patient profiles. & Implemented AI-Assist to achieve this functionality. & \href{https://github.com/users/Inreet-Kaur/projects/6/views/2?pane=issue&itemId=103747848&issue=PKALXI%7CRapidCare%7C381}{\#381}, \href{https://github.com/users/Inreet-Kaur/projects/6/views/2?pane=issue&itemId=103750671&issue=PKALXI%7CRapidCare%7C384}{\#384} \\
    \hline
    Usability Testing & Add disclaimers for predictions. & Implemented clear disclaimers when providing treatment plans and diagnosis predictions. & \href{https://github.com/users/Inreet-Kaur/projects/6/views/2?pane=issue&itemId=104780091&issue=PKALXI%7CRapidCare%7C426}{\#426} \\
    \hline
    Usability Testing, Peers & Add confirmation modals and appropriate feedback messages. & Implemented confirmation modals and toast messages to provide feedback to the user. & \href{https://github.com/users/Inreet-Kaur/projects/6/views/2?pane=issue&itemId=104780373&issue=PKALXI%7CRapidCare%7C427}{\#427} \\
    \hline
\end{longtable}

\begin{longtable}{| p{0.1\textwidth} | p{0.4\textwidth} | p{0.3\textwidth} | p{0.1\textwidth} |}
    \caption{Changes for MG and MIS} \\
    \hline
    \textbf{Feedback Source} & \textbf{Feedback Item} & \textbf{Response} & \textbf{Issue} \\
    \hline
    \endfirsthead
    \hline
    \textbf{Feedback Source} & \textbf{Feedback Item} & \textbf{Response} & \textbf{Issue} \\
    \hline
    \endhead
    \hline
    \endfoot
    TA & SoftArchitec: Quality Information: Some secrets are not secrets and should be updated. & Updated to clarify the information. & \href{https://github.com/users/Inreet-Kaur/projects/2/views/1?pane=issue&itemId=104763444&issue=PKALXI%7CRapidCare%7C419}{\#419} \\
    \hline
    TA & DetDesDoc: EnoughToBuild: Vague and ambiguous wording. & Clarified wording for better understanding. & \href{https://github.com/users/Inreet-Kaur/projects/2/views/1?pane=issue&itemId=104763999&issue=PKALXI%7CRapidCare%7C420}{\#420} \\
    \hline
    TA & CI/CD Infrastructure: Not working, actions failing. & CI/CD fully set up with no issues. & \href{https://github.com/users/Inreet-Kaur/projects/2/views/1?pane=issue&itemId=104764205&issue=PKALXI%7CRapidCare%7C421}{\#421} \\
    \hline
    TA & LO SpecMath: No specific feedback provided. & Already formalized to the best of our ability. & \href{https://github.com/users/Inreet-Kaur/projects/2/views/1?pane=issue&itemId=104764560&issue=PKALXI%7CRapidCare%7C422}{\#422} \\
    \hline
    TA & LO ProbSolutions: No specific feedback provided. & Already completed to the best of our ability. & \href{https://github.com/users/Inreet-Kaur/projects/2/views/1?pane=issue&itemId=104764784&issue=PKALXI%7CRapidCare%7C423}{\#423} \\
    \hline
    TA & LO Explores: No specific feedback provided. & Already completed to the best of our ability. & \href{https://github.com/users/Inreet-Kaur/projects/2/views/1?pane=issue&itemId=104765036&issue=PKALXI%7CRapidCare%7C424}{\#424} \\
    \hline
    Peer Review & Module Guide: Secrets for patient model and administrator model do not cover data validation, consistency rules, and any internal logic hidden from other modules. & Minor changes added; the main secret is mentioned, and data validation is inferred as a function. & \href{https://github.com/PKALXI/RapidCare/issues/249}{\#249} \\
    \hline
    Peer Review & MG: Insufficient Detail on Relationships Between Modules. & Provided details in the network sections, including HTTP and sockets. & \href{https://github.com/PKALXI/RapidCare/issues/250}{\#250} \\
    \hline
    Peer Review & MIS: Missing Details in Assumptions for Prediction Modules. & Updated assumptions to include edge cases, such as handling missing or invalid data in the input chart. & \href{https://github.com/PKALXI/RapidCare/issues/251}{\#251} \\
    \hline
    Peer Review & MIS: Missing Error Handling Details in Broker Module. & Added details on how the Broker Module handles module-level failures. & \href{https://github.com/PKALXI/RapidCare/issues/252}{\#252} \\
    \hline
    Peer Review & MG: Lack of Integration Details for Administrator Account Management. & Included a description of the request/response structure for the CRUD operations and specified validation requirements for API calls. & \href{https://github.com/PKALXI/RapidCare/issues/253}{\#253} \\
    \hline
    Design Changes & NA & MG and MIS updated to reflect the updated design, including added components like the AI-Assist module, etc. & \href{https://github.com/users/Inreet-Kaur/projects/2/views/9?pane=issue&itemId=104552607&issue=PKALXI%7CRapidCare%7C405}{\#405} \\
    \hline
\end{longtable}

\subsection{VnV Plan and Report}

Changes to VnV Plan and Report along with the feedback, response, and associated issues can be found in the table below: 

\begin{longtable}{| p{0.1\textwidth} | p{0.4\textwidth} | p{0.3\textwidth} | p{0.1\textwidth} |}
    \caption{Changes for VnV Plan} \\
    \hline
    \textbf{Feedback Source} & \textbf{Feedback Item} & \textbf{Response} & \textbf{Issue} \\
    \hline
    \endfirsthead
    \hline
    \textbf{Feedback Source} & \textbf{Feedback Item} & \textbf{Response} & \textbf{Issue} \\
    \hline
    \endhead
    \hline
    \endfoot
    TA & Content: Broken entry in references. & Reference issue fixed. & \href{https://github.com/users/Inreet-Kaur/projects/2/views/8?sliceBy%5Bvalue%5D=TA+Feedback+Issues&pane=issue&itemId=93054421&issue=PKALXI%7CRapidCare%7C190}{\#190} \\
    \hline
    TA & Spelling, grammar, and style: Break into paragraphs such that one paragraph discusses one topic. & Formatting fixed. & \href{https://github.com/users/Inreet-Kaur/projects/2/views/8?sliceBy%5Bvalue%5D=TA+Feedback+Issues&pane=issue&itemId=93054661&issue=PKALXI%7CRapidCare%7C191}{\#191} \\
    \hline
    TA & Plan: VnV Reviews issues. & Updated to remove mutation testing and include a checklist. & \href{https://github.com/users/Inreet-Kaur/projects/2/views/8?sliceBy%5Bvalue%5D=TA+Feedback+Issues&pane=issue&itemId=93054750&issue=PKALXI%7CRapidCare%7C194}{\#194} \\
    \hline
    TA & System Tests for Functional Requirements are specific: Issues in input and error messages. & Provided concrete inputs and error messages. & \href{https://github.com/users/Inreet-Kaur/projects/2/views/8?sliceBy%5Bvalue%5D=TA+Feedback+Issues&pane=issue&itemId=93056576&issue=PKALXI%7CRapidCare%7C197}{\#197} \\
    \hline
    TA & Tests for Nonfunctional Requirements are specific: Vague tests. & Updated tests to include specific details. & \href{https://github.com/users/Inreet-Kaur/projects/2/views/8?sliceBy%5Bvalue%5D=TA+Feedback+Issues&pane=issue&itemId=93056734&issue=PKALXI%7CRapidCare%7C198}{\#198} \\
    \hline
    TA & Nondynamic testing used as necessary: Details missing for static testing. & Added details for static testing and fixed errors. & \href{https://github.com/users/Inreet-Kaur/projects/2/views/8?sliceBy%5Bvalue%5D=TA+Feedback+Issues&pane=issue&itemId=93060932&issue=PKALXI%7CRapidCare%7C200}{\#200} \\
    \hline
    Peer Review & General Information: Lacks clarity in its objectives. & Updated to emphasize the critical nature of safety and security in healthcare applications. & \href{https://github.com/PKALXI/RapidCare/issues/137}{\#137} \\
    \hline
    Peer Review & Usability survey: Does not include specific and open-ended questions. & Usability survey updated to include tailored questions to collect data on various design components. & \href{https://github.com/PKALXI/RapidCare/issues/138}{\#138} \\
    \hline
    Peer Review & Implementation Verification Plan: Does not provide clear criteria for identifying "critical sections." & Updated to include a checklist to identify critical sections. & \href{https://github.com/PKALXI/RapidCare/issues/139}{\#139} \\
    \hline
    Peer Review & Tests for functional requirements: Lacks details on error messages. & Updated output to include specific details in error messages. & \href{https://github.com/PKALXI/RapidCare/issues/140}{\#140} \\
    \hline
    Peer Review & Static Testing Procedures: Lacks a structured approach. & Static Testing Procedures updated to include a structured approach and a checklist. & \href{https://github.com/PKALXI/RapidCare/issues/141}{\#141} \\
    \hline
    Peer Review & Software Validation Plan: Lacks specific criteria for validating the software against stakeholder expectations. & Updated to include specific criteria to validate software. & \href{https://github.com/PKALXI/RapidCare/issues/142}{\#142} \\
    \hline
    Design Changes & NA & VnV Plan updated to add tests for AI-Assist, updated existing tests to meet the final implementation design. & \href{https://github.com/users/Inreet-Kaur/projects/2/views/9?pane=issue&itemId=104552151&issue=PKALXI%7CRapidCare%7C404}{\#404} \\
    \hline
\end{longtable}

\begin{longtable}{| p{0.1\textwidth} | p{0.4\textwidth} | p{0.3\textwidth} | p{0.1\textwidth} |}
    \caption{Changes for VnV Report} \\
    \hline
    \textbf{Feedback Source} & \textbf{Feedback Item} & \textbf{Response} & \textbf{Issue} \\
    \hline
    \endfirsthead
    \hline
    \textbf{Feedback Source} & \textbf{Feedback Item} & \textbf{Response} & \textbf{Issue} \\
    \hline
    \endhead
    \hline
    \endfoot
    Peer Review & VnV Report: Verification of Nonfunctional Requirement 1 is missing. & Results from usability testing included. & \href{https://github.com/PKALXI/RapidCare/issues/357}{\#357} \\
    \hline
    Peer Review & Report has grammar and spelling issues. & Updated to fix grammar and spelling. & \href{https://github.com/PKALXI/RapidCare/issues/358}{\#358} \\
    \hline
    Peer Review & VnV Report: Insufficient Explanation for Deviations from the VnV Plan. & Both VnV Plan and Report updated to include changes due to updated requirements. Any deviations are fully explained. & \href{https://github.com/PKALXI/RapidCare/issues/359}{\#359} \\
    \hline
    Peer Review & VnV Report: Shallow Safety and Security Testing. & Updated safety and security requirements to include specific criteria and match updated requirements. & \href{https://github.com/PKALXI/RapidCare/issues/360}{\#360} \\
    \hline
    Peer Review & VnV Report: Revision history not updated. & Revision history updated to include details. & \href{https://github.com/PKALXI/RapidCare/issues/361}{\#361} \\
    \hline
    Peer Review & VnV Report: Undefined Pass/Fail Criteria for Voice-to-Text Transcription. & Tests updated to include specifics. Also covered as part of unit testing. & \href{https://github.com/PKALXI/RapidCare/issues/362}{\#362} \\
    \hline
    Design Changes & NA & VnV Report updated to add test results for AI-Assist, and test results for updated tests from VnV Plan to meet the final implementation design. & \href{https://github.com/users/Inreet-Kaur/projects/2/views/9?pane=issue&itemId=104552151&issue=PKALXI%7CRapidCare%7C404}{\#404} \\
    \hline
\end{longtable}

\section{Challenge Level and Extras}

\subsection{Challenge Level}

The challenge level for the project is \textbf{General} as agreed upon by the course instructor. This classification perfectly reflects the project's scope and complexity.

\subsection{Extras}

The extras that were took by this project are usability testing and user manual.
In usability testing, the participants were given some task instructions to test the system. Post task, they were required to fill a survey to rate their experience with the system. The participants also gave some suggestions for future improvements in the system.
Additionally, a user manual is a technical document that is provided to assist people in using this project. It contains detailed description of each feature of the system as well as instructions on how to use it. 

\section{Design Iteration (LO11 (PrototypeIterate))}

The journey from the first version to the final version was a driven by the iterative feedback from the supervisor, TA, peers, other stakeholders, and the professor. 

Initially, the goals were defined in the problem statement and the initial set of requirements was laid out in SRS. After meeting with the supervisor and hospital tour, we updated our goals and some requirements to introduce new features that will distinguish this project with the EHR that is being used currently in the healthcare industry. We showcased our system before the Proof of Concept demo to our supervisor to gather her feedback and following that we included the necessary features in the system. Post that, we showcased our POC demo to the TA and the supervisor to explain how will we mitigate the potential risks associated with the modules. After gathering the feedback from the supervisor, TA and the professor, we made a list of the features that we will be priortizing before Rev 0 demo. The initial UI design was prepared in figma for both MG and MIS and the tests were prepared in VnVPlan. We also performed the usability testing with the initial version with our peers and our supervisor. To prepare for Rev 0 demo, we completed most of the requirements stated in SRS and also performed an initial testing of the UI. Following that, we showcased our system before the Rev 0 demo to our supervisor and adding two additional functionalities, i.e., Referrals and Prescriptions to further enhance the features of our system. Then we did our Rev 0 demo with the TA and professor. Based on their feedback, we added another functionality which is AI-assistance to help healthcare professionals load and query patient data. We also re-structured unit tests along with improving the accuracy for transcription and classification services. After improving the accuracy, we showcased our project to the supervisor and did usability testing with the updated system. Following her feedback, we added the disclaimer to the UI in the footer section. Therefore, the final product is a result of iterative process as explained above.        

\section{Design Decisions (LO12)}

\plt{Reflect and justify your design decisions.  How did limitations,
 assumptions, and constraints influence your decisions?  Discuss each of these
 separately.}

 \section{Economic Considerations (LO23)}

 There is a clear market for our product, RapidCare. The healthcare industry, particularly in Ontario, is facing an overwhelming documentation burden due to a shortage of family doctors, affecting over 2.5 million patients. Our solution automates healthcare documentation through voice-to-text transcription and ML-based diagnosis and medication suggestions, targeting hospitals and clinics to improve efficiency and reduce wait times. \\
 
 
 \noindent
 Marketing the product would involve outreach to healthcare networks and hospital systems, emphasizing the benefits of reduced documentation overhead, improved patient throughput, and clinician satisfaction. Usability testing and comprehensive user documentation are already part of the project deliverables to aid adoption. \\
 
 
 \noindent
 We have estimated the cost to produce and maintain a market-ready version at $\$500$ per month. Each subscription will place us in a net positive, making the business model financially sustainable from the first sale. \\
 
 \noindent
 Our current strategy involves direct outreach to hospital IT departments, leveraging existing connections to secure early adopters and drive initial growth. The potential user base includes doctors, nurses, and administrative staff across healthcare institutions, particularly those involved in clinical documentation. Given Ontario’s reported shortage and the size of its healthcare system, there are thousands of potential users.
 
 \section{Reflection on Project Management (LO24)}
 
 \subsection{How Does Your Project Management Compare to Your Development Plan}
 
 We followed the development plan closely. Team meetings were held weekly, with agendas, minutes, and action items tracked via GitHub issues. Microsoft Teams was used for communication, and GitHub Projects was used for documentation, code reviews, and task tracking. The technologies planned were all implemented as expected.
 
 \subsection{What Went Well?}
 
 \begin{itemize}
     \item Clear team roles and responsibilities helped maintain accountability.
     \item GitHub Projects provided an effective workflow.
     \item Code review and peer feedback ensured quality control.
 \end{itemize}
 
 \subsection{What Went Wrong?}
 
 \begin{itemize}
     \item The variability in healthcare documentation between institutions posed elicitation challenges.
     \item Integration testing was limited during early phases and could have been started earlier.
 \end{itemize}
 
 \subsection{What Would You Do Differently Next Time?}
 
 \begin{itemize}
     \item Increase stakeholder engagement throughout mid-development milestones.
 \end{itemize}
 
 \section{Reflection on Capstone}
 
 \subsection{Which Courses Were Relevant}
 
 The following courses were highly relevant to our capstone project:
 \begin{itemize}
     \item SFWRENG 3DB3 - Databases.
     \item ENGINEER 3PX3 - Integrated Engineering Design Project 3.
     \item SFWRENG 3A04 - Software Design III.
     \item SFWRENG 4HC3 - Human-Computer Interfaces.
     \item SFWRENG 4AL3 - Applications of Machine Learning.
 \end{itemize}
 
 \subsection{Knowledge/Skills Outside of Courses}
 
 We had to acquire several skills beyond what was covered in our coursework:
 \begin{itemize}
     \item OAuth 2.0 and secure authentication practices.
     \item Domain-specific knowledge about healthcare documentation and medical terminology.
 \end{itemize}

\end{document}