% THIS DOCUMENT IS FOLLOWS THE VOLERE TEMPLATE BY Suzanne Robertson and James Robertson
% ONLY THE SECTION HEADINGS ARE PROVIDED
%
% Initial draft from https://github.com/Dieblich/volere
%
% Risks are removed because they are covered by the Hazard Analysis
\documentclass[12pt]{article}

\usepackage{booktabs}
\usepackage{tabularx}
\usepackage{hyperref}
\hypersetup{
    bookmarks=true,         % show bookmarks bar?
      colorlinks=true,      % false: boxed links; true: colored links
    linkcolor=red,          % color of internal links (change box color with linkbordercolor)
    citecolor=green,        % color of links to bibliography
    filecolor=magenta,      % color of file links
    urlcolor=cyan           % color of external links
}
\usepackage{amsmath, mathtools}
\usepackage{amsfonts}
\usepackage{amssymb}
\usepackage{graphicx}
\usepackage{colortbl}
\usepackage{xr}
\usepackage{longtable}
\usepackage{xfrac}
\usepackage{float}
\usepackage{siunitx}
\usepackage{caption}
\usepackage{pdflscape}
\usepackage{afterpage}
\usepackage{float}
\usepackage{fullpage}
\usepackage[round]{natbib}

\newcommand{\lips}{\textit{Insert your content here.}}

\input{../Comments}
\input{../Common}

% For easy change of table widths
\newcommand{\colZwidth}{1.0\textwidth}
\newcommand{\colAwidth}{0.13\textwidth}
\newcommand{\colBwidth}{0.82\textwidth}
\newcommand{\colCwidth}{0.1\textwidth}
\newcommand{\colDwidth}{0.05\textwidth}
\newcommand{\colEwidth}{0.8\textwidth}
\newcommand{\colFwidth}{0.17\textwidth}
\newcommand{\colGwidth}{0.5\textwidth}
\newcommand{\colHwidth}{0.28\textwidth}

% Used so that cross-references have a meaningful prefix
\newcounter{defnum} %Definition Number
\newcommand{\dthedefnum}{GD\thedefnum}
\newcommand{\dref}[1]{GD\ref{#1}}
\newcounter{datadefnum} %Datadefinition Number
\newcommand{\ddthedatadefnum}{DD\thedatadefnum}
\newcommand{\ddref}[1]{DD\ref{#1}}
\newcounter{theorynum} %Theory Number
\newcommand{\tthetheorynum}{TM\thetheorynum}
\newcommand{\tref}[1]{TM\ref{#1}}
\newcounter{tablenum} %Table Number
\newcommand{\tbthetablenum}{TB\thetablenum}
\newcommand{\tbref}[1]{TB\ref{#1}}
\newcounter{assumpnum} %Assumption Number
\newcommand{\atheassumpnum}{A\theassumpnum}
\newcommand{\aref}[1]{A\ref{#1}}
\newcounter{goalnum} %Goal Number
\newcommand{\gthegoalnum}{GS\thegoalnum}
\newcommand{\gsref}[1]{GS\ref{#1}}
\newcounter{stretchgoalnum} %Stretch Goal Number
\newcommand{\sgthestretchgoalnum}{STG\thestretchgoalnum}
\newcommand{\sgref}[1]{STG\ref{#1}}
\newcounter{instnum} %Instance Number
\newcommand{\itheinstnum}{IM\theinstnum}
\newcommand{\iref}[1]{IM\ref{#1}}
\newcounter{reqnum} %Requirement Number
\newcommand{\rthereqnum}{R\thereqnum}
\newcommand{\rref}[1]{R\ref{#1}}
\newcounter{nfrnum} %NFR Number
\newcommand{\rthenfrnum}{NFR\thenfrnum}
\newcommand{\nfrref}[1]{NFR\ref{#1}}
\newcounter{lcnum} %Likely change number
\newcommand{\lthelcnum}{LC\thelcnum}
\newcommand{\lcref}[1]{LC\ref{#1}}
\newcounter{ulcnum} %Unlikely change number
\newcommand{\ltheulcnum}{ULC\theulcnum}
\newcommand{\ulcref}[1]{ULC\ref{#1}}

\begin{document}

\title{Software Requirements Specification for \progname: RapidCare} 
\author{\authname}
\date{\today}
	
\maketitle

~\newpage

\pagenumbering{roman}

\tableofcontents

~\newpage

\section*{Revision History}

\begin{tabularx}{\textwidth}{p{3cm}p{2cm}X}
\toprule {\textbf{Date}} & {\textbf{Version}} & {\textbf{Notes}}\\
\midrule
12-10-2025 & 1.0 & Rev 0\\
06-01-2025 & 1.1 & Pranav Updated: TA feedback parts (Removed implementation details, fixed wordings, Traceability Matrix)\\
19-03-2025 & 1.2 & Template Update \\
\bottomrule
\end{tabularx}

~\\

~\newpage

\section{Introduction}

\subsection{Purpose of Document} \label{sec_PurposeOfDocument}
The purpose of this document is to provide a comprehensive description of the requirements for a software application that aims to streamline the healthcare documentation process aimed to be run as a web application. This document will be used in as a contract in a sense between the team and the client who intends to use this application. This document will allow for an in-depth description of the software's functionality, performance, and other non-functional requirements. Additionally, it will outline common use-cases under which the software will be used. This will in turn provide a direction to the developers such that they will be empowered to creating the right product as this document will contain various stakeholders' requirements. Along with development direction, this document will be a direct reference for all of the stakeholders to understand the product's scope, functionality, and limitations.

\section{Purpose of the Project}

\subsection{User Business}
Ontario is facing an extreme shortage of family doctors, with the number of patients without one jumping by 600,000 to 2.5 million which is a growing number [1]. This situation is only to get worse as predicted by the Ontario Medical Association [2]. As a result, people find themselves going to the ER with coughs and colds and flooding the ER causing massive wait times which ends in patients even resulting in leaving without being seen [3]. A massive part of the wait time is due to the overhead of documentation tasks. Doctors, healthcare professionals, and support staff find themselves spending most of their time on documentation which overall slows the pipeline of patients tremendously.

\subsection{Goals of the Project}
% Gurleen copy from PS


\section{Stakeholders}

\subsection{Client}
The primary client for this project are healthcare institutions such as hospitals and clinics. These institutions will benefit from the project as it will help in streamlining the documentation process, reducing administrative overhead, and enhancing the quality of patient care.

\subsection{Customer}
The primary customer of the project are healthcare professionals, specifically those involved in patient documentation, including doctors, nurses, and other clinical administrative staff. The project will provide a user-friendly interface that allows for efficient data entry and retrieval, as well as accurate transcription of patient interactions. The ystem will minimize the time spent on documentation, allowing them to focus more on patient care.

\subsection{Other Stakeholders}
\begin{itemize}
  \item \textbf{Society:} Society is a major stakeholder in this project. The healthcare institutions such as hospitals and clinics are a part of users group within the society that will be benefitted with improved efficiency. Additionally, they will be able to manage their staff better which in turn help in managing cost. Moreover, this project will help patients to get improved access to healthcare which will benenfit the society. 
  \item \textbf{Regulatory Bodies:} These are  the organizations that set standards for healthcare technology and data protection. They are concerned with compliance to regulations such as PIPEDA and other relevant laws.
\end{itemize}

\subsection{Hands-On Users of the Project}
\begin{itemize}
  \item\textbf{Healthcare Professionals:} These include the hospital staff such as Doctors, nurses etc, who will use the system.
  \item\textbf{Healthcare Network Employees:} These are the people employed in the organization who keeps records of all hospital facilities, their staff members and authenticates the healthcare professionals so that they are able to use the system. 
\end{itemize}

\subsection{Personas}
\begin{itemize}
    \item \textbf{Healthcare Professional: Dr. Virat} - Dr. Virat works in the emergency department at Brampton Hospital. He frequently uses EHR systems to document clinical notes and manage patient records. However, he feels that most of his time is spent recording notes. He wishes for a system that could speed up this process and allow him to focus more on direct patient care and serve more patients throughout the day.
    \item \textbf{Nurse: Anushka} - Anushka is a registered nurse who relies on the system to access patient information quickly and update records during patient triage. She often feel streesed due to increased patient volume and long hours of shift due to inefficent staff maangmement at her hospital.
    \item \textbf{Patient: John} - John is a working professional who recently visited the emergency department for a sprained ankle. He had to wait 13 hours in the emergency room due to high patient volume and inefficient documentation process. He expects the system that reduce wait times .
\end{itemize}

\subsection{Priorities Assigned to Users}
\begin{itemize}
    \item \textbf{High Priority:} Healthcare professionals have the highest priority as they are direct users of the system and are directly involved in patient care and operational efficiency.
    \item \textbf{Medium Priority:} Healthcare network administrators have medium priority as they are responsible for managing user access and ensuring data integrity but do not interact with the system as frequently as healthcare professionals.
    \item \textbf{Low Priority:} All other stakeholders including patients and regulatory bodies have low priority. While they benefit from the system's functionality, they are not direct users and rely on healthcare professionals to manage their information.
\end{itemize}

\subsection{User Participation}
User participation is vital for the design, development, and acceptance testing of the system. Users like healthcare professionals can provide feedback during the design and testing phases to ensure the system meets their critical needs. This will help in gathering insights on usability and functionality, allowing for iterative improvements to the system.

\subsection{Maintenance Users and Service Technicians}
System administrators and IT staff will be responsible for maintaining the system. They would be responsible to provide assistance to the users and troubleshoot any technical issues that may arise ensuring that the system remains functional. 

\section{Mandated Constraints}
\subsection{Solution Constraints}
\lips
\subsection{Implementation Environment of the Current System}
\lips
\subsection{Partner or Collaborative Applications}
\lips
\subsection{Off-the-Shelf Software}
\lips
\subsection{Anticipated Workplace Environment}
\lips
\subsection{Schedule Constraints}
\lips
\subsection{Budget Constraints}
\lips
\subsection{Enterprise Constraints}
\lips

\section{Naming Conventions and Terminology}
\subsection{Glossary of All Terms, Including Acronyms, Used by Stakeholders
involved in the Project}
\lips

\section{Relevant Facts And Assumptions}
\subsection{Relevant Facts}
\lips
\subsection{Business Rules}
\lips
\subsection{Assumptions}
\lips

\section{The Scope of the Work}
\subsection{The Current Situation}
\lips
\subsection{The Context of the Work}
\lips
\subsection{Work Partitioning}
\lips
\subsection{Specifying a Business Use Case (BUC)}
\lips

\section{Business Data Model and Data Dictionary}
\subsection{Business Data Model}
\lips
\subsection{Data Dictionary}
\lips

\section{The Scope of the Product}
\subsection{Product Boundary}
\lips
\subsection{Product Use Case Table}
\lips
\subsection{Individual Product Use Cases (PUC's)}
\lips

\section{Functional Requirements}
\subsection{Functional Requirements}
\lips

\section{Look and Feel Requirements}
\subsection{Appearance Requirements}
\lips
\subsection{Style Requirements}
\lips

\section{Usability and Humanity Requirements}
\subsection{Ease of Use Requirements}
\lips
\subsection{Personalization and Internationalization Requirements}
\lips
\subsection{Learning Requirements}
\lips
\subsection{Understandability and Politeness Requirements}
\lips
\subsection{Accessibility Requirements}
\lips

\section{Performance Requirements}
\subsection{Speed and Latency Requirements}
\lips
\subsection{Safety-Critical Requirements}
\lips
\subsection{Precision or Accuracy Requirements}
\lips
\subsection{Robustness or Fault-Tolerance Requirements}
\lips
\subsection{Capacity Requirements}
\lips
\subsection{Scalability or Extensibility Requirements}
\lips
\subsection{Longevity Requirements}
\lips

\section{Operational and Environmental Requirements}
\subsection{Expected Physical Environment}
\lips
\subsection{Wider Environment Requirements}
\lips
\subsection{Requirements for Interfacing with Adjacent Systems}
\lips
\subsection{Productization Requirements}
\lips
\subsection{Release Requirements}
\lips

\section{Maintainability and Support Requirements}
\subsection{Maintenance Requirements}
\lips
\subsection{Supportability Requirements}
\lips
\subsection{Adaptability Requirements}
\lips

\section{Security Requirements}
\subsection{Access Requirements}
\lips
\subsection{Integrity Requirements}
\lips
\subsection{Privacy Requirements}
\lips
\subsection{Audit Requirements}
\lips
\subsection{Immunity Requirements}
\lips

\section{Cultural Requirements}
\subsection{Cultural Requirements}
\lips

\section{Compliance Requirements}
\subsection{Legal Requirements}
\lips
\subsection{Standards Compliance Requirements}
\lips

\section{Open Issues}
\lips

\section{Off-the-Shelf Solutions}
\subsection{Ready-Made Products}
\lips
\subsection{Reusable Components}
\lips
\subsection{Products That Can Be Copied}
\lips

\section{New Problems}
\subsection{Effects on the Current Environment}
\lips
\subsection{Effects on the Installed Systems}
\lips
\subsection{Potential User Problems}
\lips
\subsection{Limitations in the Anticipated Implementation Environment That May
Inhibit the New Product}
\lips
\subsection{Follow-Up Problems}
\lips

\section{Tasks}
\subsection{Project Planning}
\lips
\subsection{Planning of the Development Phases}
\lips

\section{Migration to the New Product}
\subsection{Requirements for Migration to the New Product}
\lips
\subsection{Data That Has to be Modified or Translated for the New System}
\lips

\section{Costs}
\lips
\section{User Documentation and Training}
\subsection{User Documentation Requirements}
\lips
\subsection{Training Requirements}
\lips

\section{Waiting Room}
\lips

\section{Ideas for Solution}
\lips



~\newpage

\section{References}
\begin{itemize}
  \item
  [1]N. Ireland, "Number of Ontarians without family doctor reaches 2.5 million, college says," CBC, Jul. 12, 2024. https://www.cbc.ca/news/canada/toronto/ontario-family-doctor-shortage-record-high-1.7261558
  \href{https://www.cbc.ca/news/canada/toronto/ontario-family-doctor-shortage-record-high-1.7261558}{Article on Doctor Shortage.}
  \item 
  [2]Ryan Patrick Jones, "Family doctor shortage affects every region and is getting worse, Ontario Medical Association says," CBC, Jan. 29, 2024.
  
  https://www.cbc.ca/news/canada/toronto/family-doctor-shortage-oma-1.7097935
  \href{https://www.cbc.ca/news/canada/toronto/family-doctor-shortage-oma-1.7097935}{Article on Doctor Shortage.}
  \item
  [3]"ICES | Association between waiting times and short term mortality and hospital admission after departure from the emergency department: population-based cohort study from Ontario, Canada," ICES, Jun. 14, 2023. https://www.ices.on.ca/publications/journal-articles/association-between-waiting-times-and-short-term-mortality-and-hospital-admission-after-departure-from-the-emergency-department-population-based-cohort-study-from-ontario-canada/ (accessed Oct. 12, 2024).
  \href{https://www.ices.on.ca/publications/journal-articles/association-between-waiting-times-and-short-term-mortality-and-hospital-admission-after-departure-from-the-emergency-department-population-based-cohort-study-from-ontario-canada/}{Article on ER patients leaving without being seen.}
\end{itemize}

\newpage{}
\section*{Appendix --- Reflection}

The information in this section will be used to evaluate the team members on the
graduate attribute of Lifelong Learning.  Please answer the following questions:

\begin{enumerate}
  \item What went well while writing this deliverable?
  
  This document has let us build more on the rough ideas we had brainstormed initially. While going through the outline of this document, we were able to provide the functional and non-functional requirements of the system. It also made us better understand the detailed procedure of automation using the use-case diagram. 

  \item What pain points did you experience during this deliverable, and how did
  you resolve them?

  There are obstacles in any team project that must be overcome for it to proceed successfully. To ensure seamless operations, we had to develop a strategy for contributions. We must plan a template that aligns with our project to make sure we have all the requirements for the system. We also needed to create a schedule to contribute to the template and review each other's work in the best way possible.
  
  \item How many of your requirements were inspired by speaking to your
  client(s) or their proxies (e.g. your peers, stakeholders, potential users)?

  The stakeholder didn't explicitly contribute to the requirements section of the project, however, the stakeholder did give us her insight on the working of the software and EMR apps in clinics and hospitals. This helped us to refine our usage scenario diagram and has been a great source of help in the progress of the project.

  \item Which of the courses you have taken, or are currently taking, will help
  your team to be successful with your capstone project.

  In terms of courses, we have taken our requirements course (SFWRENG 3RA3), system design course (SFWRENG 3A04), as well as our testing course (SFWRENG 3S03). These courses have helped us develop the requirement documents but also prepared us in the sense that we are able to look into the future to what the testing and design strategy should look like. These courses have played a pivotal role in our understanding and creation of our documents. 

  \item What knowledge and skills will the team collectively need to acquire to
  successfully complete this capstone project?  Examples of possible knowledge
  to acquire include domain specific knowledge from the domain of your
  application, or software engineering knowledge, mechatronics knowledge or
  computer science knowledge.  Skills may be related to technology, or writing,
  or presentation, or team management, etc.  You should look to identify at
  least one item for each team member.

  As different abilities are added to the overall development plan, the project progresses more quickly thanks to the diversified knowledge of the team members. Technical expertise in data processing and integration would be very beneficial for this capstone project. In addition, time management abilities are required to make sure that everyone is moving at the same rate and that they are informed about each other's work to prevent backlogs. Additionally, being able to work on backend programming with an understanding of different server-side technologies, like Python, Java, and React.js, will be helpful in creating a dynamic database that secures patient data. 

  \item For each of the knowledge areas and skills identified in the previous
  question, what are at least two approaches to acquiring the knowledge or
  mastering the skill?  Of the identified approaches, which will each team
  member pursue, and why did they make this choice?

  One can use a variety of resources to learn Java from Spring and Python from Flask to become effective in backend programming. With these resources, developers can gain practical experience by building web applications that offer them flexibility. Using LeetCode to practice coding skills is an additional strategy. This allows users to modify the difficulty of the problems and proceed with their solution. Moreover, setting goals and taking regular pauses between tasks might help with time management and allow one to work more effectively. It's critical that each member of the team grasp every talent for everyone to be moving at the same speed. This is because it's a fantastic chance to learn and implement it in practical situations. 

\end{enumerate}

\end{document}