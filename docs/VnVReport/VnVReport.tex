\documentclass[12pt, titlepage]{article}

\usepackage{booktabs}
\usepackage{tabularx}
\usepackage{hyperref}
\hypersetup{
    colorlinks,
    citecolor=black,
    filecolor=black,
    linkcolor=red,
    urlcolor=blue
}
\usepackage[round]{natbib}

\input{../Comments}
\input{../Common}

\begin{document}

\title{Verification and Validation Report: \progname} 
\author{\authname}
\date{\today}
	
\maketitle

\pagenumbering{roman}

\section{Revision History}

\begin{tabularx}{\textwidth}{p{3cm}p{2cm}X}
\toprule {\bf Date} & {\bf Version} & {\bf Notes}\\
\midrule
Date 1 & 1.0 & Notes\\
Date 2 & 1.1 & Notes\\
\bottomrule
\end{tabularx}

~\newpage

\section{Symbols, Abbreviations and Acronyms}

\renewcommand{\arraystretch}{1.2}
\begin{tabular}{l l} 
  \toprule		
  \textbf{symbol} & \textbf{description}\\
  \midrule 
  T & Test\\
  \bottomrule
\end{tabular}\\

\wss{symbols, abbreviations or acronyms -- you can reference the SRS tables if needed}

\newpage

\tableofcontents

\listoftables %if appropriate

\listoffigures %if appropriate

\newpage

\pagenumbering{arabic}

This document ...

\section{Functional Requirements Evaluation} \label{section:3} 

\subsection{Add a document to the database} \label{section:3.1}

This subsection covers FR1, FR4, and FR8 from of the \href{https://github.com/Inreet-Kaur/capstone/blob/main/docs/SRS/SRS.pdf}{SRS document} by testing that the system is able to add a document to the database only when a valid input is provided.

\begin{enumerate}

  \item{test-FR1,4,8-1} \label{test-FR1,4,8-1}
  
  Initial State:

  Input:

  Expected Output:

  Actual Output:

  Result:


  \item{test-FR1,4,8-2} \label{test-FR1,4,8-2}

  Initial State:

  Input:

  Expected Output:

  Actual Output:

  Result:

\end{enumerate}

\subsection{Remove a document to the database} \label{section:3.2}

This subsection covers FR2, FR5, and FR9 from of the \href{https://github.com/Inreet-Kaur/capstone/blob/main/docs/SRS/SRS.pdf}{SRS document} by testing that the system is able to remove a document to the database only when a valid input idetifier is provided.

\begin{enumerate}

  \item{test-FR2,5,9-1} \label{test-FR2,5,9-1}
  
  Initial State:

  Input:

  Expected Output:

  Actual Output:

  Result:


  \item{test-FR2,5,9-2} \label{test-FR2,5,9-2}

  Initial State:

  Input:

  Expected Output:

  Actual Output:

  Result:

\end{enumerate}

\subsection{Update a document to the database} \label{section:3.3}

This subsection covers FR3, FR6, FR10, and FR11 from of the \href{https://github.com/Inreet-Kaur/capstone/blob/main/docs/SRS/SRS.pdf}{SRS document} by testing that the system is able to update a document to the database only when a valid input idetifier is provided.

\begin{enumerate}

  \item{test-FR3,6,10,11-1} \label{test-FR3,6,10,11-1}
  
  Initial State:

  Input:

  Expected Output:

  Actual Output:

  Result:


  \item{test-FR3,6,10,11-2} \label{test-FR3,6,10,11-2}

  Initial State:

  Input:

  Expected Output:

  Actual Output:

  Result:

\end{enumerate}

\subsection{Login for valid/invalid credentials} \label{section:3.4}

This subsection covers FR7 from of the \href{https://github.com/Inreet-Kaur/capstone/blob/main/docs/SRS/SRS.pdf} {SRS document} by testing that the system is able to allow to access the database only when a valid input credentials is provided.

\begin{enumerate}

  \item{test-FR7-1} \label{test-FR7-1}
  
  Initial State:

  Input:

  Expected Output:

  Actual Output:

  Result:


  \item{test-FR7-2} \label{test-FR7-2}

  Initial State:

  Input:

  Expected Output:

  Actual Output:

  Result:

\end{enumerate}

\subsection{Voice-to-text-transcription check} \label{section:3.5}

This subsection covers FR7 from of the \href{https://github.com/Inreet-Kaur/capstone/blob/main/docs/SRS/SRS.pdf} {SRS document} by testing that the system is able to transcribe audio data to the written text.

\begin{enumerate}

  \item{test-FR11-1} \label{test-FR11-1}
  
  Initial State:

  Input:

  Expected Output:

  Actual Output:

  Result:


  \item{test-FR11-2} \label{test-FR11-2}

  Initial State:

  Input:

  Expected Output:

  Actual Output:

  Result:

\end{enumerate}

\subsection{Validate output of correct diagnosis and medication} \label{section:3.6}

This subsection covers FR12 and FR13 from of the \href{https://github.com/Inreet-Kaur/capstone/blob/main/docs/SRS/SRS.pdf} {SRS document} by testing that the system is able to create predictions on the diagnosis and medicines with a high accuracy and confidence.

\begin{enumerate}

  \item{test-FR12,13-1} \label{test-FR12,13-1}
  
  Initial State:

  Input:

  Expected Output:

  Actual Output:

  Result:

\end{enumerate}


\subsection{Validate input data for models} \label{section:3.7}

This subsection covers FR12, FR13, and IR5 from of the \href{https://github.com/Inreet-Kaur/capstone/blob/main/docs/SRS/SRS.pdf} {SRS document} by testing that the system is able to validate the input data in the charts such that it may be inputted into the prediction module.

\begin{enumerate}

  \item{test-FR12,13,IR5-1} \label{test-FR12,13,IR5-1}
  
  Initial State:

  Input:

  Expected Output:

  Actual Output:

  Result:

  \item{test-FR12,13,IR5-2} \label{test-FR12,13,IR5-2}
  
  Initial State:

  Input:

  Expected Output:

  Actual Output:

  Result:

\end{enumerate}

\subsection{Verify completeness and correctmess of data} \label{section:3.8}

This subsection covers FR14 from of the \href{https://github.com/Inreet-Kaur/capstone/blob/main/docs/SRS/SRS.pdf} {SRS document} by testing that the data sent over is complete.

\begin{enumerate}

  \item{test-FR14-1} \label{test-FR14-1}
  
  Initial State:

  Input:

  Expected Output:

  Actual Output:

  Result:

\end{enumerate}

\section{Nonfunctional Requirements Evaluation} \label{section:4}

\subsection{Aesthetic and Design} \label{section:4.1}

\begin{itemize}
\item \textbf{test-AD1} \label{test-AD1} \\
\textbf{Initial State:} UI is designed and implemented. \\
\textbf{Input:} Team members and peers view the UI under normal operating conditions. \\
\textbf{Output:} Feedback collected on UI’s aesthetic appeal and simplicity. \\
\textbf{Result:} Pass \\

\item \textbf{test-AD2} \label{test-AD2} \\
\textbf{Initial State:} UI is operational and accessible to users. \\
\textbf{Input:} Users interact with the UI during routine tasks and provide feedback. \\
\textbf{Output:} Team members and peers report satisfaction with the UI design. \\
\textbf{Result:} Pass \\
\end{itemize}

\subsection{Usability} \label{section:4.2}

\begin{itemize}
\item \textbf{test-UR1} \label{test-UR1} \\
\textbf{Initial State:} System is fully available and accessible to users after a 30-minute training session. \\
\textbf{Input:} Users perform key functions (logging in, accessing records, adding entries, generating reports). \\
\textbf{Output:} Team members and peers complete tasks without assistance within 2 minutes per task. \\
\textbf{Result:} Pass \\

\item \textbf{test-UR2} \label{test-UR2} \\
\textbf{Initial State:} System is deployed and accessible. \\
\textbf{Input:} Users navigate and explore system features independently. \\
\textbf{Output:} Majority of users can locate core functions without additional guidance. \\
\textbf{Result:} Pass \\
\end{itemize}

\subsection{Performance} \label{section:4.3}

\begin{itemize}
\item \textbf{test-PR1} \label{test-PR1} \\
\textbf{Initial State:} Transcription interface open and ready. \\
\textbf{Input:} Real-time voice input provided by healthcare professionals. \\
\textbf{Output:} Real-time transcription displayed within a 2-second delay. \\
\textbf{Result:} Pass \\
\end{itemize}

\subsection{Operational} \label{section:4.4}

\begin{itemize}
\item \textbf{test-OR1} \label{test-OR1} \\
\textbf{Initial State:} System is live and connected to monitoring software. \\
\textbf{Input:} 7-day operational period with intermittent load testing. \\
\textbf{Output:} Team members and peers monitored uptime consistently maintained at 99.9\% or above. \\
\textbf{Result:} Pass \\

\item \textbf{test-OR2} \label{test-OR2} \\
\textbf{Initial State:} System in operational use. \\
\textbf{Input:} Users access the system over a 7-day period under normal and peak loads. \\
\textbf{Output:} Team members and peers confirm that uptime meets standards without significant interruptions. \\
\textbf{Result:} Pass \\
\end{itemize}

\subsection{Maintainability} \label{section:4.5}

\begin{itemize}
\item \textbf{test-MR1} \label{test-MR1} \\
\textbf{Initial State:} System running on the latest version with recent update logs. \\
\textbf{Input:} Regular software update applied for bug fixes and improvements. \\
\textbf{Output:} System successfully applies updates without impacting stability. \\
\textbf{Result:} Pass \\

\item \textbf{test-MR2} \label{test-MR2} \\
\textbf{Initial State:} Previous version of the system with identified bugs or issues. \\
\textbf{Input:} Apply updates addressing known issues. \\
\textbf{Output:} System functions as expected with resolved issues and no new errors introduced. \\
\textbf{Result:} Pass \\
\end{itemize}

\subsection{Security} \label{section:4.6}

\begin{itemize}
\item \textbf{test-SR1} \label{test-SR1} \\
\textbf{Initial State:} System with live patient data encryption protocols active. \\
\textbf{Input:} Security audit test performed on the system. \\
\textbf{Output:} No vulnerabilities detected; all data remains encrypted in transit and at rest. \\
\textbf{Result:} Pass \\

\item \textbf{test-SR2} \label{test-SR2} \\
\textbf{Initial State:} System fully operational with access logs enabled. \\
\textbf{Input:} Simulate unauthorized access attempts. \\
\textbf{Output:} Unauthorized attempts are blocked; access logs capture details. \\
\textbf{Result:} Pass \\
\end{itemize}

\subsection{Cultural} \label{section:4.7}

\begin{itemize}
\item \textbf{test-CR1} \label{test-CR1} \\
\textbf{Initial State:} System operational in default language (English). \\
\textbf{Input:} User selects an alternate language from the settings menu. \\
\textbf{Output:} System displays all content in the selected language without loss of functionality. \\
\textbf{Result:} Fail \\
\end{itemize}

\subsection{Legal} \label{section:4.8}

\begin{itemize}
\item \textbf{test-LR1} \label{test-LR1} \\
\textbf{Initial State:} System is fully functional and contains patient records. \\
\textbf{Input:} Conduct a compliance audit with PIPEDA and other relevant data protection standards. \\
\textbf{Output:} System passes all compliance checks with no exceptions. \\
\textbf{Result:} Pass \\

\item \textbf{test-LR2} \label{test-LR2} \\
\textbf{Initial State:} System storing and transmitting patient data over a network. \\
\textbf{Input:} Monitor data handling and transfer processes during operation. \\
\textbf{Output:} All patient data is handled in compliance with regulations, without any unauthorized access or data breaches. \\
\textbf{Result:} Pass \\
\end{itemize}

\subsection{Scalability} \label{section:4.9}

\begin{itemize}
\item \textbf{test-S1} \label{test-S1} \\
\textbf{Initial State:} System deployed on a test server environment capable of scaling horizontally. \\
\textbf{Input:} Simulate an increasing number of concurrent users. \\
\textbf{Output:} System maintains consistent performance and response times without degradation. \\
\textbf{Result:} Pass \\
\end{itemize}


\section{Comparison to Existing Implementation}	

This section will not be appropriate for every project.

\section{Unit Testing}

\section{Changes Due to Testing}

\wss{This section should highlight how feedback from the users and from 
the supervisor (when one exists) shaped the final product.  In particular 
the feedback from the Rev 0 demo to the supervisor (or to potential users) 
should be highlighted.}

\section{Automated Testing}
		
\section{Trace to Requirements}
		
\section{Trace to Modules}		

\section{Code Coverage Metrics}

\bibliographystyle{plainnat}
\bibliography{../../refs/References}

\newpage{}
\section*{Appendix --- Reflection}

The information in this section will be used to evaluate the team members on the
graduate attribute of Reflection.

\input{../Reflection.tex}

\begin{enumerate}
  \item What went well while writing this deliverable?
  This document has let us strengthen the understanding about unit testing of all functional requirements as well as verification of all non-functional qualities of the system. While going through the outline of the document, we were able to conduct unit testing and compare it with the existing implementation. It also made us better execute the detailed procedure of system testing along with performing validation and verification of the system. 

  \item What pain points did you experience during this deliverable, and how did you resolve them?
  There are obstacles in any team project that must be overcome for it to proceed successfully. To ensure seamless operations, we had to develop a strategy for contributions. Since we already had the list of tests that were needed to be conducted for both functional and non-functional requirements, we divided them equally among all the team members. We also needed to create a schedule to contribute to the template and review each other's work in the best way possible. 

  \item Which parts of this document stemmed from speaking to your client(s) or a proxy (e.g. your peers)? Which ones were not, and why?
  The client's feedback helped us to understand the changes due to testing the requirements. Before Rev 0 demo, the team gave a demo to the supervisor about the system pretending the supervisor as the healthcare professional using the system. The supervisor then gave us some possible ways that the healthcare professional would like to use the system. This helped us to improve some features of the system such as whether prescription of a patient is being saved or not. If yes, then is it being stored within the system or on local machine. Furthermore, the section such as unit testing of all functional requirements stemmed from the collective ideas of our peers. We discussed different test cases that each requirement could possibly have along with how they will be implemented. In this way, we got diverse test cases and all team members were able to contribute especially in this critical section of the report.

  \item In what ways was the Verification and Validation (VnV) Plan different
  from the activities that were actually conducted for VnV?  If there were
  differences, what changes required the modification in the plan?  Why did
  these changes occur?  Would you be able to anticipate these changes in future
  projects?  If there weren't any differences, how was your team able to clearly
  predict a feasible amount of effort and the right tasks needed to build the
  evidence that demonstrates the required quality?  (It is expected that most
  teams will have had to deviate from their original VnV Plan.)
  VnV Plan was little bit different from the activities that were actually conducted and recorded in VnV Report. This is because VnV Plan includes one extra functional requirement test which needs to be deleted as the team has decided to not pursue with that functional requirement. Therefore, the team will modify the VnV Plan to remove that requirement. This change occured as we went through every detail of our system and it turns out that functional requirement is not necessary for the system to accomplish its goal. Yes, we will be able to anticipate thus change in the future projects as going through every detail of the project review the necessary and non-necessary features of the product.  

\end{enumerate}

\end{document}