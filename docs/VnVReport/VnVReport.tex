\documentclass[12pt, titlepage]{article}

\usepackage{booktabs}
\usepackage{tabularx}
\usepackage{hyperref}
\hypersetup{
    colorlinks,
    citecolor=black,
    filecolor=black,
    linkcolor=red,
    urlcolor=blue
}
\usepackage[round]{natbib}

\input{../Comments}
\input{../Common}

\begin{document}

\title{Verification and Validation Report: \progname} 
\author{\authname}
\date{\today}
	
\maketitle

\pagenumbering{roman}

\section{Revision History}

\begin{tabularx}{\textwidth}{p{3cm}p{2cm}X}
\toprule {\bf Date} & {\bf Version} & {\bf Notes}\\
\midrule
Date 1 & 1.0 & Notes\\
Date 2 & 1.1 & Notes\\
\bottomrule
\end{tabularx}

~\newpage

\section{Symbols, Abbreviations and Acronyms}

\renewcommand{\arraystretch}{1.2}
\begin{tabular}{l l} 
  \toprule		
  \textbf{symbol} & \textbf{description}\\
  \midrule 
  T & Test\\
  \bottomrule
\end{tabular}\\

\wss{symbols, abbreviations or acronyms -- you can reference the SRS tables if needed}

\newpage

\tableofcontents

\listoftables %if appropriate

\listoffigures %if appropriate

\newpage

\pagenumbering{arabic}

This document ...

\section{Functional Requirements Evaluation} \label{section:3} 

\subsection{Add a document to the database} \label{section:3.1}

This subsection covers FR1, FR4, and FR8 from of the \href{https://github.com/Inreet-Kaur/capstone/blob/main/docs/SRS/SRS.pdf}{SRS document} by testing that the system is able to add a document to the database only when a valid input is provided.

\begin{enumerate}

  \item{test-FR1,4,8-1} \label{test-FR1,4,8-1}
  
  Initial State:

  Input:

  Expected Output:

  Actual Output:

  Result:


  \item{test-FR1,4,8-2} \label{test-FR1,4,8-2}

  Initial State:

  Input:

  Expected Output:

  Actual Output:

  Result:

\end{enumerate}

\subsection{Remove a document to the database} \label{section:3.2}

This subsection covers FR2, FR5, and FR9 from of the \href{https://github.com/Inreet-Kaur/capstone/blob/main/docs/SRS/SRS.pdf}{SRS document} by testing that the system is able to remove a document to the database only when a valid input idetifier is provided.

\begin{enumerate}

  \item{test-FR2,5,9-1} \label{test-FR2,5,9-1}
  
  Initial State:

  Input:

  Expected Output:

  Actual Output:

  Result:


  \item{test-FR2,5,9-2} \label{test-FR2,5,9-2}

  Initial State:

  Input:

  Expected Output:

  Actual Output:

  Result:

\end{enumerate}

\subsection{Update a document to the database} \label{section:3.3}

This subsection covers FR3, FR6, FR10, and FR11 from of the \href{https://github.com/Inreet-Kaur/capstone/blob/main/docs/SRS/SRS.pdf}{SRS document} by testing that the system is able to update a document to the database only when a valid input idetifier is provided.

\begin{enumerate}

  \item{test-FR3,6,10,11-1} \label{test-FR3,6,10,11-1}
  
  Initial State:

  Input:

  Expected Output:

  Actual Output:

  Result:


  \item{test-FR3,6,10,11-2} \label{test-FR3,6,10,11-2}

  Initial State:

  Input:

  Expected Output:

  Actual Output:

  Result:

\end{enumerate}

\subsection{Login for valid/invalid credentials} \label{section:3.4}

This subsection covers FR7 from of the \href{https://github.com/Inreet-Kaur/capstone/blob/main/docs/SRS/SRS.pdf} {SRS document} by testing that the system is able to allow to access the database only when a valid input credentials is provided.

\begin{enumerate}

  \item{test-FR7-1} \label{test-FR7-1}
  
  Initial State:

  Input:

  Expected Output:

  Actual Output:

  Result:


  \item{test-FR7-2} \label{test-FR7-2}

  Initial State:

  Input:

  Expected Output:

  Actual Output:

  Result:

\end{enumerate}

\subsection{Voice-to-text-transcription check} \label{section:3.5}

This subsection covers FR7 from of the \href{https://github.com/Inreet-Kaur/capstone/blob/main/docs/SRS/SRS.pdf} {SRS document} by testing that the system is able to transcribe audio data to the written text.

\begin{enumerate}

  \item{test-FR11-1} \label{test-FR11-1}
  
  Initial State:

  Input:

  Expected Output:

  Actual Output:

  Result:


  \item{test-FR11-2} \label{test-FR11-2}

  Initial State:

  Input:

  Expected Output:

  Actual Output:

  Result:

\end{enumerate}

\subsection{Validate output of correct diagnosis and medication} \label{section:3.6}

This subsection covers FR12 and FR13 from of the \href{https://github.com/Inreet-Kaur/capstone/blob/main/docs/SRS/SRS.pdf} {SRS document} by testing that the system is able to create predictions on the diagnosis and medicines with a high accuracy and confidence.

\begin{enumerate}

  \item{test-FR12,13-1} \label{test-FR12,13-1}
  
  Initial State:

  Input:

  Expected Output:

  Actual Output:

  Result:

\end{enumerate}


\subsection{Validate input data for models} \label{section:3.7}

This subsection covers FR12, FR13, and IR5 from of the \href{https://github.com/Inreet-Kaur/capstone/blob/main/docs/SRS/SRS.pdf} {SRS document} by testing that the system is able to validate the input data in the charts such that it may be inputted into the prediction module.

\begin{enumerate}

  \item{test-FR12,13,IR5-1} \label{test-FR12,13,IR5-1}
  
  Initial State:

  Input:

  Expected Output:

  Actual Output:

  Result:

  \item{test-FR12,13,IR5-2} \label{test-FR12,13,IR5-2}
  
  Initial State:

  Input:

  Expected Output:

  Actual Output:

  Result:

\end{enumerate}

\subsection{Verify completeness and correctmess of data} \label{section:3.8}

This subsection covers FR14 from of the \href{https://github.com/Inreet-Kaur/capstone/blob/main/docs/SRS/SRS.pdf} {SRS document} by testing that the data sent over is complete.

\begin{enumerate}

  \item{test-FR14-1} \label{test-FR14-1}
  
  Initial State:

  Input:

  Expected Output:

  Actual Output:

  Result:

\end{enumerate}

\section{Nonfunctional Requirements Evaluation} \label{section:4}

\subsection{Aesthetic and Design} \label{section:4.1}

\begin{enumerate}

  \item{test-AD1} \label{test-AD1}

  Initial State:

  Input:

  Expected Output:

  Actual Output:

  Result:

  \item{test-AD2} \label{test-AD2}

  Initial State:

  Input:

  Expected Output:

  Actual Output:

  Result:

end{enumerate}

\subsection{Usability} \label{section:4.2}

\begin{enumerate}

  \item{test-UR1} \label{test-UR1}

  Initial State:

  Input:

  Expected Output:

  Actual Output:

  Result:

  \item{test-UR2} \label{test-UR2}

  Initial State:

  Input:

  Expected Output:

  Actual Output:

  Result:

end{enumerate}

\subsection{Performance} \label{section:4.3}

\begin{enumerate}

  \item{test-PR1} \label{test-PR1}

  Initial State:

  Input:

  Expected Output:

  Actual Output:

  Result:

  \item{test-PR2} \label{test-PR2}

  Initial State:

  Input:

  Expected Output:

  Actual Output:

  Result:

end{enumerate}

\subsection{Operational} \label{section:4.4}

\begin{enumerate}

  \item{test-OR1} \label{test-OR1}

  Initial State:

  Input:

  Expected Output:

  Actual Output:

  Result:

  \item{test-OR2} \label{test-OR2}

  Initial State:

  Input:

  Expected Output:

  Actual Output:

  Result:

end{enumerate}

\subsection{Maintainability} \label{section:4.5}

\begin{enumerate}

  \item{test-MR1} \label{test-MR1}

  Initial State:

  Input:

  Expected Output:

  Actual Output:

  Result:

  \item{test-MR2} \label{test-MR2}

  Initial State:

  Input:

  Expected Output:

  Actual Output:

  Result:

end{enumerate}

\subsection{Security} \label{section:4.6}

\begin{enumerate}

  \item{test-SR1} \label{test-SR1}

  Initial State:

  Input:

  Expected Output:

  Actual Output:

  Result:

  \item{test-SR2} \label{test-SR2}

  Initial State:

  Input:

  Expected Output:

  Actual Output:

  Result:

end{enumerate}

\subsection{Cultural} \label{section:4.7}

\begin{enumerate}

  \item{test-CR1} \label{test-CR1}

  Initial State:

  Input:

  Expected Output:

  Actual Output:

  Result:

  \item{test-CR2} \label{test-CR2}

  Initial State:

  Input:

  Expected Output:

  Actual Output:

  Result:

end{enumerate}

\subsection{Legal} \label{section:4.8}

\begin{enumerate}

  \item{test-LR1} \label{test-LR1}

  Initial State:

  Input:

  Expected Output:

  Actual Output:

  Result:

  \item{test-LR2} \label{test-LR2}

  Initial State:

  Input:

  Expected Output:

  Actual Output:

  Result:

end{enumerate}

\subsection{Scalability} \label{section:4.9}

\begin{enumerate}

  \item{test-S1} \label{test-S1}

  Initial State:

  Input:

  Expected Output:

  Actual Output:

  Result:

  \item{test-S2} \label{test-S2}

  Initial State:

  Input:

  Expected Output:

  Actual Output:

  Result:

end{enumerate}


\section{Comparison to Existing Implementation}	

This section will not be appropriate for every project.

\section{Unit Testing}

\section{Changes Due to Testing}

\wss{This section should highlight how feedback from the users and from 
the supervisor (when one exists) shaped the final product.  In particular 
the feedback from the Rev 0 demo to the supervisor (or to potential users) 
should be highlighted.}

\section{Automated Testing}
		
\section{Trace to Requirements}
		
\section{Trace to Modules}		

\section{Code Coverage Metrics}

\bibliographystyle{plainnat}
\bibliography{../../refs/References}

\newpage{}
\section*{Appendix --- Reflection}

The information in this section will be used to evaluate the team members on the
graduate attribute of Reflection.

\input{../Reflection.tex}

\begin{enumerate}
  \item What went well while writing this deliverable? 
  \item What pain points did you experience during this deliverable, and how
    did you resolve them?
  \item Which parts of this document stemmed from speaking to your client(s) or
  a proxy (e.g. your peers)? Which ones were not, and why?
  \item In what ways was the Verification and Validation (VnV) Plan different
  from the activities that were actually conducted for VnV?  If there were
  differences, what changes required the modification in the plan?  Why did
  these changes occur?  Would you be able to anticipate these changes in future
  projects?  If there weren't any differences, how was your team able to clearly
  predict a feasible amount of effort and the right tasks needed to build the
  evidence that demonstrates the required quality?  (It is expected that most
  teams will have had to deviate from their original VnV Plan.)
\end{enumerate}

\end{document}